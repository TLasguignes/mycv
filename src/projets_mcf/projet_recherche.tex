
\section{Projet de Recherche}

Dans les dernières décennies, la vision par ordinateur a pris de l'ampleur, notamment avec l'avènement de la recherche en robotique ou des progrès techniques dans les transports et notamment les véhicules. Les techniques de perception extéroceptives sont nécessaires pour s'adapter à un environnement variable et en évolution constante. Les capteurs extéroceptifs permettent d'obtenir des informations sur l'environnement externe du système, au contraire des capteurs prorioceptifs concentrant les mesures sur l'état interne du système.
Parmi ces systèmes de perception, les systèmes de vision basés sur des caméras et des capteurs de profondeur ont gagné en attrait. En effet, ces derniers permettent d'établir une représentation de l'environnement combinant géométrie et texture visuelle.

L'émergeance de divers types de capteurs a d'abord inspiré l'idée de combiner des caméras avec des capteurs proprioceptifs pour améliorer les systèmes de perceptions tels que les algorithmes de Simultaneous Localisation And Mapping (SLAM). Parmi ces capteurs, on peut notamment noter l'évolution des LiDAR, pour Light Detection and Ranging, une famille de capteurs utilisant des faisceaux laser pour mesurer la distance à un obstacle dans l'environnement. Les progrès effectués dans les composants électroniques ont permis au fil des années de passer d'un LiDAR "point" à un LiDAR "plan" puis un capteur permettant d'obtenir en une mesure une image à large champ de vision de l'environnement, souvent sous la forme d'un nuage de points 3D. Dans la continuité de ces progrès, on voit aujourd'hui la naissance de capteurs mêlant les technologies LiDAR et caméras RGB pour obtenir une représentation texturée visuellement et précise géométriquement. Ces capteurs permettent d'obtenir une meilleure représentation de l'environnement, éliminant les limitations associées à la présence d'une seule source d'information ou à la portée réduite des technologies utilisées précédemment. Grâce à ces progrès techniques dans les systèmes de capteurs, les algorithmes se complexifient et gagnent en fonctionnalité et en précision. Toutefois, le développement de ces algorithmes permet d'obtenir de meilleurs résultats mais augmente la difficulté d'embarquement dans des systèmes souvent limités en puissance de calcul et en énergie.

\subsection{La perception en robotique}

Pour mieux évoluer dans son environnement, un système autonome, tel un robot, nécessite un système de perception extéroceptif. Dans le cadre de mes recherches, j'ai résumé ces besoins en 3 grands axes:
\begin{itemize}
  \item La localisation du robot dans son environnement, pour pouvoir naviguer précisemment dans son environnement et savoir en tous temps où se trouve le robot,
  \item La localisation d'objets, pour connaître la pose des objets d'intérêt et pouvoir se diriger vers eux voire les manipuler,
  \item La localisation de surfaces, pour permettre une connaissance du sol pouvant être exploitée pour le déplacement.
\end{itemize}

Au cours de ma thèse, j'ai étudié chaque situation pour m'approprier et évaluer extensivement des algorithmes disponibles ou en développer de nouveaux afin de répondre aux différentes problématiques pour Talos, un robot Humanoïde utilisé dans l'équipe GEPETTO du LAAS-CNRS.

De par le contexte industriel qui était donné à mes travaux, avec des environnements composés de larges volumes avec peu de jeu de textures visuelles, je me suis attelé à l'utilisation de capteurs de profondeurs, et plus particulièrement de LiDARs. Ces capteurs de profondeurs permettent d'obtenir des données géométriques en 3 dimensions pour représenter l'environnement. Ces données sont ensuite utilisées pour extraire différentes informations de corrélation géométrique pour réaliser une tâche spécifique.

J'ai ainsi consacré mon temps à rechercher et implémenter des solutions pour permettre à un robot humanoïde de mieux évoluer dans son environnement. Pour cela, je me suis parfois référé à une représentation connue de l'environnement, et parfois à la découverte que l'on fait de cet environnement.

Je souhaite continuer ma recherche autour de la perception de l'environnement. Cette perception sera particulièrement centrée vers la segmentation des données mesurées pour identifier distinctement les différents objets et les éléments de l'environnement. Elle présentera alors plusieurs avantages. Les instances segmentées peuvent être ensuite utilisés pour localiser des objets d'intérêts pour des tâches de locomotion dirigée ou de manipulation. Elles présentent aussi un intérêt dans les systèmes de localisation puisque ces entités peuvent être utilisées comme des amers dans l'environnement, donnant des points de référence pour estimer une transformation entre la mesure actuelle et la connaissance précédente. Par exemple, le système de localisation basé LiDAR implémenté sur le robot Talos extrait des plans pour servir de repères.

\subsection{L'embarquement des solutions}

Les algorithmes que j'ai utilisés et implémentés jusqu'alors permettent par exemple d'obtenir la localisation du robot dans son environnement ou une représentation des surfaces du sol pour pouvoir planifier les contacts utiles au déplacement du robot. Cependant, la complexité et le coût combinatoire de ces algorithmes rendent difficile leur embarquement sur le robot. Il est alors nécessaire de faire évoluer ces algorithmes pour réduire la complexité ou d'améliorer le matériel pour fournir des capacités accrues, afin de ne pas assujetir l'autonomie des robots à l'utilisation de plateforme de calculs externes ajoutant aux problématiques de perception d'autres problématiques liées à la communication entre le robot et le calculateur.

Je souhaite inclure dans mes recherches un objectif de "frugalité" des algorithmes, mettant en adéquation les besoins applicatifs, les capteurs, les plateformes et les algorithmes. Ces recherches auront pour objectif premier de chercher à embarquer les algorithmes dans les systèmes robotiques.

En plus de la question de frugalité des systèmes de perception, pour un fonctionnement autonome et sécuritaire, il est nécessaire de s'assurer que les algorithmes de perception permettent d'obtenir des résultats en respectant des contraintes de latence ou de cadence, d'empreinte mémoire voire de coût énergétique.

L'embarquement des systèmes de perception recherchés demandera donc de respecter les contraintes temporelles impliquées par l'utilité et l'application ainsi que des contraintes matérielles.

\subsection{Intégration dans le laboratoire}

Ces travaux de recherches s'inscriront dans les travaux menés au Laboratoire d'Analyse et d'Architecture des Systèmes (LAAS-CNRS). Les recherches menées au LAAS-CNRS s'axent sur 4 grands axes, que sont l'Informatique, la Robotique, l'Automatique et les Micro et Nano Systèmes. Ces recherches visent à répondre aux grands défis à venir apportés par les progrès techniques dans les domaines tels que les transports, l'industrie, la santé ou les énergies. Mes travaux de recherches pourront particulièrement s'inscrire dans les travaux de l'équipe Robotique Action et Perception (RAP) du département Robotique dont les travaux s'axent sur la conception d'algorithmes pour la perception ainsi que la conception de systèmes mélant la perception et le contrôle. L'équipe pourrait notamment être intéréssée par mes recherches autour de la reconnaissance de l'environnement et l'utilisation de données 3D. De plus, l'équipe RAP s'est intéressée aux capteurs dits "intelligents", alliant des questions d'efficience énergétique, de faible latence, ainsi que les questions d'embarquement.

Je pourrais m'intégrer à court terme dans l'équipe en apportant mes compétences dans les problématiques étudiées actuellement dans l'équipe. Ces connaissances peuvent être utiles dans les problématiques de navigation référencée capteurs comme pour le déplacement de robots dans des environnements agricoles, ou dans les travaux d'asservissement visuel pour la manipulation tels que l'utilisation de capteurs de profondeur pour le positionnement d'outils pour la taille de vigne.

Au sein du bassin industriel Toulousain, de telles recherches peuvent par exemple intéresser les entreprises ayant des intérêts dans les mobilités nouvelles, telles qu'Easy Mile, ou l'évolution de l'Industrie, comme Airbus. Enfin, les travaux sur la reconnaissance de l'environnement peuvent aussi ouvrir à des collaborations internes par exemple avec l'équipe GEPETTO travaillant sur la génération de mouvement de locomotion et de manipulation.
