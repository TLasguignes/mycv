
\section{Projet d'enseignements}

Depuis le début de mes études suppérieures, je me suis intéressé à la transmission de la connaissance et à l'enseignement. Dans un premier temps dans ma pratique sportive, pendant laquelle j'ai pris part à l'encadrement de jeunes ou débutants au Balma Arc Club et au sein de l'Association Sportive de l'INSA. Puis, dès ma première année de doctorat, j'ai ajouté une dimensionnement d'enseignement à mon doctorat. J'ai ainsi pris part à l'enseignement au sein de l'INSA Toulouse puis de l'Université Toulouse III Paul Sabatier. De par ma formation originellement technique, débutée par un DUT GEII avant de rejoindre l'INSA, j'ai pu apporter un côté plus pratique à mes interventions dans divers domaines. Ainsi, j'ai pu intervenir aussi bien en électronique qu'en automatique ou en programmation. Pendant mes services d'enseignement, j'ai conservé une diversité qui me permet d'intervenir dans différentes matières, liées à ma formation tournée informatique industrielle, systèmes embarqués et automatique. Mes activités de recherche et l'équipe dans laquelle j'ai évolué m'ont aussi permis de développer des connaissances en robotique.

Au cours de mes enseignements à l'INSA pendant mon doctorat, je suis principalement intervenu dans l'encadrement des travaux dirigés et travaux pratiques, en m'impliquant dans les évaluations de certaines matières. J'ai ainsi pu observer les méthodes de constructions des enseignements. Par la suite, dans le cadre de mes contrats d'Attaché Temporaire d'Enseignement et de Recherche à l'Université et à l'INSA, j'ai eu l'opportunité d'être responsables d'enseignements. Dans ce cadre, j'ai apprécié les enseignements tels qu'ils étaient alors prodigués et fait évolué les enseignements en adaptant le contenu des cours, des exercices, des travaux pratiques et des évaluations. J'ai aussi eu l'occasion d'imaginer de nouveaux travaux pratiques pour apporter de nouveaux exemples aux étudiants. J'ai toujours cherché à apporter de nouvelles connaissances aux étudiants en avançant étape par étape, en commençant par des exercices guidés pour apprendre à monter une réflexion et ensuite étudier des travaux dans lequel ils sont moins guidés et doivent tracer eux-même leur réflexion pour arriver à un résultat fonctionnel et optimal. De plus, afin d'améliorer l'implication des étudiants dans les exercices, je me suis appliqué, et continuerai à m'appliquer, à avoir des exercices et des exemples pratiques, mélant des actions ludiques, professionnalisantes ou liant à des exemples de société pour ne pas appliquer simplement un principe mais voir que ce principe est déjà présent autour de nous. D'après moi, avoir de tels exemples aide à la compréhension des notions et au développement de certaines logiques.

Ma formation axée sur les systèmes embarqués m'a permis d'acquérir des connaissances en systèmes temps réels que j'ai ensuite entretenues et parfaites avec mes enseignements. Ces connaissances me permettront de former les étudiants sur les thématiques des systèmes temps réels, de la programmation parallèles et les problématiques environnantes. Mes compétences pourront servir aux enseignements dans le Master Ingénierie des Systèmes Temps Réels (ISTR) comme dans la formation Systèmes Robotiques et Interactifs (SRI) de l'UPSSITECH, voire dans la licence Électronique, Énergie électrique, Automatique (EEA) sur les thématiques d'Informatique Industrielle. En plus de la thématique temps réel visée par le poste, la diversité des enseignements que j'ai conservé au cours de mes services me permettra d'intervenir dans les différents domaines des systèmes embarqués. Je pourrais ainsi soutenir mes collègues et m'impliquer au besoin dans les enseignements en Automatique à Evènements Discrets et Informatique Industrielle. 

De par mon doctorat en robotique, j'ai acquis des connaissances qui m'ont permis de donner des cours introductifs à la robotique. Ces mêmes connaissances pourraient être utilisées pour des enseignements en robotique. Dans ce dernier cadre, je pourrais m'impliquer pour la relance du master Automatique-Robotique (AURO) dans un effort commun avec mes collègues. Je pense qu'une formation, de type Master, orientée autour de la robotique reste nécessaire dans le paysage universitaire français et Toulousain. Dans ce cadre je suis prêt à m'impliquer dans les formations axées notamment sur le temps réel mais aussi sur la perception 3D ou d'autres enseignements où mes compétences peuvent être utiles.

Les différents enseignements sur lesquels je suis intervenu m'ont permis de voir différentes méthodes de pédagogie. Souvent axées autour de notions appliquées à des exercices, j'ai pu observer les avantages de certaines méthodes de travail, comme par exemple les travaux dirigés en groupes autonomes afin d'ouvrir les étudiants à tracer leur guide, sous la surveillance de l'encadrant. Au cours de mon cursus, j'ai eu l'occasion de rencontrer d'autres méthodes innovantes, telles que des classes inversées, de l'apprentissage par problèmes, de la pédagogie active et de l'autoévaluation régulière. Je pense personnellement que la pédagogie traditionnelle n'est pas adaptée à tous les enseignements ni à tous les étudiants, il est donc nécessaire de faire évoluer les méthodes pédagogiques pour que l'apprentissage soit plus efficace. Je suis ouvert à toutes les pédagogies qui pourront permettre aux étudiants à progresser et apprendre efficacement. Je suis aussi prêt à m'impliquer dans les discussions qui auraient lieu sur l'évolution des méthodes pédagogiques pour que le corps enseignant évolue ensemble et progresse, comme nos étudiants apprennent.

Dans le cadre de mes enseignements, j'ai aussi eu l'occasion de travailler avec des étudiants en situation atypiques, tels qu'un étudiant diagnostiqué TSA, une étudiante en situation de handicap moteur ou des étudiants dyslexiques ou dysorthographiques. Pour la plupart, leurs situations atypiques ne présentaient de contraintes dans leurs études que par rapport aux méthodes d'enseignements et d'évaluation actuelles. Par ces expériences, et par une expérience plus personnelle avec une personne en situation de handicap, j'ai été sensibilisé à la problématique de l'inclusion dans les études suppérieures. J'ai essayé dans mes enseignements de m'adapter aux situations atypiques sans pour autant diminuer la qualité de l'enseignement et espère pouvoir m'impliquer encore dans l'inclusivité des formations et des locaux pour ne pas empêcher des étudiants à apprendre à cause de leur handicap.

Enfin, au delà des enseignements et de leur évolution, je suis évidemment prêt à prendre les responsabilités d'intérêt général au sein du département EEA que les tâches soient administratives ou techniques. Ayant déjà encadré des stagiaires, du côté professionnel comme du côté enseignement, je compte m'investir encore dans l'encadrement de stagiaires, en proposant des sujets de stages alliant dans différentes proportions les thématiques de la recherche en vision par ordinateur et du développement et de l'optimisation d'algorithme. Je souhaite aussi m'impliquer dans l'encadrement et l'évaluation des stagiaires du côté enseignement.
