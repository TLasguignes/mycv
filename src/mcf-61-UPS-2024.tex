%% start of file `template.tex'.
%% Copyright 2006-2013 Xavier Danaux (xdanaux@gmail.com).
%
% This work may be distributed and/or modified under the
% conditions of the LaTeX Project Public License version 1.3c,
% available at http://www.latex-project.org/lppl/.


\documentclass[11pt,a4paper,sans]{moderncv}         % possible options include font size ('10pt', '11pt' and '12pt'), paper size ('a4paper', 'letterpaper', 'a5paper', 'legalpaper', 'executivepaper' and 'landscape') and font family ('sans' and 'roman')

% moderncv themes
\moderncvstyle{classic}                             % style options are 'casual' (default), 'classic', 'oldstyle' and 'banking'
\moderncvcolor{blue}                                % color options 'blue' (default), 'orange', 'green', 'red', 'purple', 'grey' and 'black'
%\renewcommand{\familydefault}{\sfdefault}          % to set the default font; use '\sfdefault' for the default sans serif font, '\rmdefault' for the default roman one, or any tex font name
%\nopagenumbers{}                                   % uncomment to suppress automatic page numbering for CVs longer than one page

% character encoding
\usepackage[utf8]{inputenc}                         % if you are not using xelatex ou lualatex, replace by the encoding you are using
%\usepackage{CJKutf8}                               % if you need to use CJK to typeset your resume in Chinese, Japanese or Korean

% adjust the page margins
%\usepackage[scale=0.75]{geometry}
%\usepackage[top=1.9cm, bottom=1.9cm, left=1.9cm, right=1.9cm]{geometry}
\usepackage[scale=.85]{geometry}
%\setlength{\hintscolumnwidth}{2.cm}                % if you want to change the width of the column with the dates
%\setlength{\makecvtitlenamewidth}{10cm}            % for the 'classic' style, if you want to force the width allocated to your name and avoid line breaks.
                                                    % be careful though, the length is normally calculated to avoid any overlap with your personal info; use this at your own typographical risks...

%** WHY ?
\usepackage{xparse}
%**

\usepackage{bibentry}
\usepackage{makecell}                               % for multiline cells
\usepackage{wrapfig}                                % for wraping figures
\nobibliography*

\bibliographystyle{my_plain}

\usepackage{enumitem}
\usepackage{scrextend}
\setlist{nolistsep}

% A custom version of the \cventry command that supports large itemized lists
% inside argument #7 (the custom cvitemize lists should be used!)
\newcommand*{\cventrylong}[7][.25em]{%
  \begin{tabular}{@{}p{\hintscolumnwidth}@{\hspace{\separatorcolumnwidth}}p{\maincolumnwidth}@{}}%
    \raggedleft\hintstyle{#2} &{%
        {\bfseries#3}%
        \ifthenelse{\equal{#4}{}}{}{, {\slshape#4}}%
        \ifthenelse{\equal{#5}{}}{}{, #5}%
        \ifthenelse{\equal{#6}{}}{}{, #6}%
    }%
  \end{tabular}%
  \begin{addmargin}[\hintscolumnwidth+\separatorcolumnwidth]{0em}%
    {\small#7}%
  \end{addmargin}%
  \par\addvspace{#1}}
% A custom version of the itemize environment that sets the appropriate left
% margin for use inside \cventylong
\newlist{cvitemize}{itemize}{1}
\setlist[cvitemize]{label=\labelitemi,%
leftmargin=\hintscolumnwidth+\separatorcolumnwidth+\labelwidth+\labelsep}

\usepackage{etoolbox}
\newtoggle{myrefs}
\newcommand{\myref}[1]{\iftoggle{myrefs}{\textbf{#1}}{#1}}

\name{Thibaud}{Lasguignes}
\title{Docteur en robotique humano\"ide et vision par ordinateur}            % optional, remove / comment the line if not wanted
\address{2 All\'ee Elise Deroche}{R\'esidence Lindbergh, G2-202}{31400 Toulouse} % optional, remove / comment the line if not wanted; the "postcode city" and and "country" arguments can be omitted or provided empty
\phone[mobile]{+33~(6)~26~36~28~13}                   % optional, remove / comment the line if not wanted
% \phone[fixed]{+2~(345)~678~901}                     % optional, remove / comment the line if not wanted
% \phone[fax]{+3~(456)~789~012}                       % optional, remove / comment the line if not wanted
\email{thibaud.lasguignes@sfr.fr}                     % optional, remove / comment the line if not wanted
\social[github]{TLasguignes}
% \homepage{www.johndoe.com}                          % optional, remove / comment the line if not wanted
%\photo[5cm]{../figures/DSC00009.jpeg}                % optional, remove / comment the line if not wanted; '64pt' is the height the picture must be resized to, 0.4pt is the thickness of the frame around it (put it to 0pt for no frame) and 'picture' is the name of the picture file
% \quote{Some quote}                                  % optional, remove / comment the line if not wanted

\newcommand{\items}{\item \hspace{2mm}}

% to show numerical labels in the bibliography (default is to show no labels); only useful if you make citations in your resume
%\makeatletter
%\renewcommand*{\bibliographyitemlabel}{\@biblabel{\arabic{enumiv}}}
%\makeatother
%\renewcommand*{\bibliographyitemlabel}{[\arabic{enumiv}]}% CONSIDER REPLACING THE ABOVE BY THIS

% bibliography with mutiple entries
%\usepackage{multibib}
%\newcites{book,misc}{{Books},{Others}}
%----------------------------------------------------------------------------------
%            content
%----------------------------------------------------------------------------------
\begin{document}
%\begin{CJK*}{UTF8}{gbsn}                          % to typeset your resume in Chinese using CJK
%-----       resume       ---------------------------------------------------------

\makecvtitle
\vspace*{-1cm}

\section{Formations}
\cventry
{2023}
{Doctorat en Robotique}
{INSA de Toulouse et LAAS-CNRS, \`a~Toulouse (31)}
{"Reconnaissance d'objets visant \`a la locomotion et \`a la manipulation par un robot humano\"ide dans un environnement industriel"}
{supervis\'e par Olivier Stasse}
{
  Soutenue le 31 Ao\^ut 2023 devant le jury compos\'e de :\\
  \renewcommand{\arraystretch}{2}
  \begin{tabular}{l@{\hskip .5in}l@{\hskip .5in}l}
    Olivier Aycard         & \makecell{Ma\^itre de Conf\'erence\\Universit\'e Grenoble Alpes} & Rapporteur                \\
    David Filliat          & \makecell{Professeur\\ENSTA Paris}                               & Pr\'esident et Rapporteur \\
    Jean-Emmanuel Deschaud & \makecell{Charg\'e de Recherche\\Mines Paris}                    & Examinateur               \\
    Ariane Herbulot        & \makecell{Ma\^itre de Conf\'erence\\Universit\'e Paul Sabatier Toulouse III}  & Examinateur               \\
    Nicolas Mansard        & \makecell{Directeur de Recherche\\LAAS-CNRS}                     & Examinateur               \\
    Olivier Stasse         & \makecell{Directeur de Recherche\\LAAS-CNRS}                     & Directeur de Th\`ese      \\
  \end{tabular}
}
%
\cventry
{2019}
{Dipl\^ome d'ing\'enieur INSA, Sp\'ecialit\'e Automatique Electronique}
{INSA de Toulouse (31)}
{orient\'e Syst\`emes Informatiques Embarqu\'es Critiques}
{}
{
  "Reconnaissance et localisation d’objets par vision embarqu\'ee (cam\'eras RGB ou RGB-D) sur un robot humano\"ide pour des t\^aches de vissage et de per\c cage", supervis\'e par Olivier Stasse au LAAS-CNRS
}
%
\cventry
{2016}
{DUT G\'enie Electrique et Informatique Industrielle}
{IUT `A' Paul Sabatier}
{\`a Toulouse (31)}
{(major)}
{}
%
\cventry
{2014}
{Baccalaur\'eat Scientifique option math\'ematiques}
{Lyc\'ee G\'en\'eral et Technologique Jean-Fran\c cois Champollion}
{\`a Figeac (46)}
{(mention Bien)}
{}
%
\vspace*{0.5cm}
\cventry
{Langues}
{Fran\c cais}
{langue maternelle}
{}
{}
{}
%
\cventry
{}
{Anglais}
{Niveau B2}
{}
{}
{}
%
\cventry{Informatique}
{Logiciels}
{cmake, git, ROS, PCL, Open3D}
{}
{}
{}
%
\cventry
{}
{Langage}
{C/C++, python, bash, Matlab, VHDL, Langage d'assemblage, LADDER}
{}
{}
{}
%

\section{Exp\'eriences Professionnelles}
\cventry
{Depuis 2023}
{Attach\'e Temporaire d'Enseignement et de Recherche}
{Institut National des Sciences Appliqu\'ees de Toulouse (31)}
{}
{}
{
  R\'ealisation d'enseignement au sein du d\'epartement de G\'enie Electrique et Informatique de l'INSA Toulouse sur un total de 192h \'equivalent TD.
}
%
\cventry
{2022 -- 2023}
{Attach\'e Temporaire d'Enseignement et de Recherche}
{Universit\'e Paul Sabatier Toulouse III (31)}
{}
{}
{
  R\'ealisation d'enseignement au sein du d\'epartement Electronique, Energie \'electrique et Automatique de la Facult\'e Sciences et Ing\'enierie sur un total de 192h \'equivalent TD.
}
%
\cventry
{2019 -- 2021}
{Doctorant Contractuel Charg\'e d'Enseignement}
{INSA de Toulouse (31)}
{}
{}
{
  R\'ealisation d'enseignements au sein du d\'epartement G\'enie Electrique et Informatique de l'Institut National des Sciences Appliqu\'ees de Toulouse sur deux ans, soit sur un total de 128 heures.
}
%
\cventry
{2019}
{Stagiaire}
{LAAS-CNRS}
{Toulouse (31)}
{"Reconnaissance et localisation d'objets par vision embarqu\'ee sur un robot humano\"ide"}
{
  D\'eveloppement d'un syst\`eme visant \`a la localisation d'objets en utilisant des cam\'eras RGB-D, pour fournir \`a un robot la pose de l'objet \`a manipuler en utilisant des m\'ethodes de r\'ealignement de nuages de points.
  Objets suppos\'es pos\'es sur une table et leurs mod\`eles connus.
}
%
%
\cventry
{2018}
{Stagiaire}
{Technology and Strategy}
{Munich, Allemagne}
{Testeur en v\'ehicule dans le cadre d'un projet BMW - Bosch sur une aide au parking automatique}
{
  Conduite de tests en v\'ehicules pour \'evaluer les capacit\'es du syst\`eme d\'evelopp\'e, relever les failles et permettre son am\'elioration.
}
%
%
\cventry
{2016}
{Stagiaire}
{Laboratoire d'Ing\'enierie des Syst\'emes Biologiques et des Proc\'ed\'es}
{Toulouse (31)}
{}
{
  Analyse de code et d\'eveloppement sous Matlab dans le but d'acc\'el\'erer un syst\'eme d'optimisation d'exp\'eriences microbiologiques.
}
%

\section{Activit\'es d'Enseignement}
J'ai réalisé un total de 445h d'enseignement équivalent TD couvrant les différents aspects des systèmes embarqués et pour différents niveaux. Mes interventions ont eu lieu autour de la programmation sur différents langages, l'informatique industrielle et les systèmes temps réels, mais aussi en automatique discrète et en électronique pour le traitement du signal.\\
Je me suis impliqué dans l'évaluation des étudiants dans le cadre de différentes matières, allant de la surveillance des examens à l'écriture des sujets et la correction des copies. Dans le cadre de mes postes ATER à l'Université Toulouse III Paul Sabatier et à l'INSA Toulouse, j'ai été ammené à remodeler le contenu de matières, en retravaillant voire changeant le contenu des enseignements, des travaux dirigés et des travaux pratiques. Ces matières concernait différents niveaux d'études allant de la première année au M1.\\
Dans le cadre de mes contrats ATER, j'ai aussi été emmené à accompagner des étudiants lors de leurs stages de fin d'études (niveau M2). Dans ce même cadre, je détaille dans mes Activités d'Encadrement les encadrement de stagiaires que j'ai eu l'occasion d'effectuer pendant mon doctorat.\\
Les enseignements que j'ai effectué sont résumés par domaine dans le tableau suivant et détaillés ci-apprès.\\
\begin{center}
  \vspace{-1em}
  \renewcommand{\arraystretch}{1.5}
  \begin{tabular}{|c|c|c|c|c|}
    \hline
    ~Domaine~                       & ~Cadre~        & ~Institution~ & ~\makecell{Volume horaire\\eq. TD}~ & ~Niveau ou équivalent~ \\
    \hline
    \hline
    Programmation~                  & ~ATER et DCCE~ & ~UPS et INSA~ & ~228,12h~                             & ~L3 et M1~             \\
    \hline
    Informatique Industrielle~      & ~ATER~         & ~UPS~         & ~20h~                                 & ~L1 et L2~             \\
    \hline
    ~Programmation Orientée Objets~ & ~ATER~         & ~INSA~        & ~12,83h~                              & ~M1~                   \\
    \hline
    Programmation Parallèle~        & ~ATER~         & ~UPS~         & ~15h~                                 & ~M1~                   \\
    \hline
    Systèmes Temps Réel~            & ~ATER et DCCE~ & ~UPS et INSA~ & ~71h~                                 & ~M1~                   \\
    \hline
    Automatique Discrète~           & ~ATER et DCCE~ & ~UPS et INSA~ & ~66h~                                 & ~L2, L3 et M1~         \\
    \hline
    Electronique~                   & ~ATER et DCCE~ & ~UPS et INSA~ & ~16,5h~                               & ~L2 et L3~             \\
    \hline
    Robotique et ROS~               & ~ATER et DCCE~ & ~UPS et INSA~ & ~16h~                                 & ~M1~                   \\
    \hline
  \end{tabular}
  \vspace{.5em}
\end{center}
\\
%
\cventrylong
{Depuis 2023}
{Attach\'e Temporaire d'Enseignement et de Recherche}
{Institut National des Sciences Appliqu\'ees de Toulouse (31)}
{D\'epartement de G\'enie Electrique et Informatique}
{}
{
  \textbf{Syst\`emes Temps R\'eel} (Printemps 2024)\\
  Enseignement de 22h en TP pour les 4e ann\'ee AE.\\
  L'objectif est de donner aux \'etudiants les notions de Syst\`emes Temps R\'eel.
  Les \'etudiants apprennent \`a mettre en place les notions vues en cours \`a travers la gestion de la communication avec une base mobile dont le contr\^ole a \'et\'e fourni int\'egrant des contraintes temps r\'eel sur les ordres re\c cus.\\
  %
  \textbf{Programmation Orient\'ee Objets}\\
  Enseignement de 19,25h en TP pour les 4e ann\'ee AE-SE.\\
  L'objectif est de donner des notions de programmation orient\'ee objets en langage C++ \`a travers des exercices guid\'es et un projet.
  Les \'etudiants sont d'abord amen\'es \`a \'etudier les diff\'erentes notions des objets \`a travers la cr\'eation de classe, l'h\'eritage et le polymorphisme.
  Par la suite, il leur est demand\'e de r\'ealiser le projet de leur choix autour d'un microcontr\^oleur ESP8266 et de diff\'erents capteurs et actionneurs dont ils ont besoin, en mettant en place les notions vues en cours pour structurer leur programme.\\
  %
  \textbf{Programmation langage C}\\
  Enseignement de 6,25h en TD et 11h en TP pour les 3e ann\'ee IMACS.\\
  L'objectif est de mettre en place les notions de programmation en langage C.
  Les \'etudiants sont amen\'es \`a \'ecrire diff\'erents programmes permettant de d\'ecouvrir les diff\'erentes notions et syntaxes du C.\\
  %
  \textbf{Algorithmique et programmation ADA}\\
  Enseignement de 51,25h en TD et 43,5h en TP pour les 1\`ere et 2\`eme ann\'ee.\\
  L'objectif est de donner des notions de d\'eveloppement d'algorithme et de r\'ealisation en programmation ADA.\\
  Les \'etudiants de 1\`ere ann\'ee apprennent les bases de l'\'ecriture d'algorithmes et de programmes \`a travers des travaux pratiques visuels utilisant un simulateur d'avion ou de la modification d'image simple.\\
  Les \'etudiants de 2\`eme ann\'ee stabilisent et approfondissent les notions \`a travers le d\'eveloppement d'algorithmes ludiques mettant en place des notions plus complexes telles que la r\'ecursivit\'e ou les pointeurs.\\
  %
  \textbf{Algorithmique et programmation Python} (Printemps 2024)\\
  Enseignement de 3,75h en CM, 2,5h en TD et 30,25h en TP pour les 2\`eme ann\'ee FAS option ``Construction''.\\
  L'objectif est de donner les notions de programmation et d'algorithmique aux \'etudiants \`a travers du d\'eveloppement en python.\\
  Les \'etudiants sont amen\'es \`a comprendre l'utilisation des listes et cha\^ines de caract\`es ainsi que l'utilisation de boucles dans leurs programmes \`a travers divers exercices et probl\`emes math\'ematiques simples.\\
  %
}
%
\cventrylong
{2022 -- 2023}
{Attach\'e Temporaire d'Enseignement et de Recherche}
{Universit\'e Paul Sabatier Toulouse III (31)}
{Facult\'e Sciences et Ing\'enierie}
{D\'epartement Electronique, Energie \'electrique et Automatique}
{
  \textbf{Robotique Mobile et utilisation de ROS}\\
  Enseignement de 24h en TP pour les Master 2 Automatique et Robotique.\\
  L'objectif \'etait d'apprendre la structuration des architectures robotiques autour du middleware ROS.
  Les \'etudiants apprenaient l'utilisation du middleware en simulation pour faire \'evoluer des robots dans un environnement en utilisant les syst\`emes disponibles.
  Ils devaient aussi d\'evelopper leur propre n\oe ud de contr\^ole permettant de diriger un robot mobile Tiago afin de suivre une cible d\'etect\'ee avec une cam\'era RGB.\\
  %
  \textbf{Techniques et Impl\'ementation de M\'ethodes Num\'eriques}\\
  Enseignement de 58h en TP pour les Master 1 du d\'epartement EEA.\\
  L'objectif \'etait d'apprendre la programmation en langage C en d\'eveloppant des codes appliquant des m\'ethodes num\'eriques.
  Les \'etudiants d\'eveloppaient en C des programmes allant de la r\'esolution d'\'equation du second degr\'e aux int\'egrations par parties et interpolations lin\'eaires.\\
  %
  \textbf{Outils pour la Commande de Syst\`emes Parall\`eles}\\
  Enseignement de 10h en CM pour les Master 1 ISTR-AURO.\\
  L'objectif \'etait de donner aux \'etudiants les notions n\'ecessaires \`a la programmation parall\`eles en langage C.
  Les notions \'etudi\'ees allaient de la th\'eorie des processus et de la programmation concurrente au partage de ressource et de donn\'ees.
  Les explications se voulaient g\'en\'eriques et utilisaient la librairie \emph{pthread} pour donner des exemples.
  Les \'etudiants ont ensuite \'et\'e amen\'es en TD et en TP \`a \'etudier diff\'erents probl\`emes de partage de ressource et \`a mettre en place des solutions adapt\'ees en utilisant les notions vues en cours.\\
  %
  \textbf{Syst\`emes Temps R\'eel}\\
  Enseignement de 8h en CM, 10h en TD et 24h en TP pour les 2\`eme ann\'ee SRI \`a l'UPSSITECH.\\
  L'objectif \'etait de mettre en place les notions de syst\`eme temps r\'eel, d'ordonnancement de t\^aches et de criticit\'e des t\^aches.
  Les \'etudiants apprenaient \`a mettre en place les diff\'erentes notions \`a travers des analyses de cahier des charges pour mod\'eliser la commande r\'ealisable, l'analyse de code prenant en compte les temps de calculs pour v\'erifier l'ordonnancabilit\'e.
  Ils ont ensuite mis en pratique les notions vues en r\'ealisant un g\'en\'erateur de signaux sans notions de temps r\'eel et des g\'en\'erateurs de signaux avec diff\'erentes m\'ethodes appliquant les notions vues en cours et ont observ\'e les effets de l'ordonnancement dans le cas de syst\`eme surcharg\'es.\\
  %
  \textbf{Techniques de Mises en \OE uvre pour les Syst\`emes \`a Ev\`enements Discrets}\\
  Enseignement de 18h en TP pour les Master 1 ISTR-AURO.\\
  L'objectif \'etait de d\'evelopper les notions de mise en \oe uvre de syst\`emes \`a \'ev\'enements discrets en partant d'une mod\'elisation pour aller sur la programmation.
  Les \'etudiants \'etaient amen\'es \`a analyser un cahier des charges, mod\'eliser une commande r\'epondant et programmer la mod\'elisation en langage ST, IL, C ou VHDL selon la m\'ethode et la plateforme.
  Les \'etudiants ont ensuite d\^u d\'evelopper en C un syst\`eme automatisant le proc\'ed\'e de programmation.
  Leurs programmes devaient r\'ecup\'erer les informations de la commande et \'ecrire le code n\'ecessaire \`a la mise en place du syst\`eme dans un langage d\'efini.\\
  %
  \textbf{Automatique \`a Evenements Discrets}\\
  Enseignement de 60h en TP pour les Licence 3 du d\'epartement EEA.\\
  L'objectif \'etait de mettre en place les notions d'automatique \`a \'ev\'enements discrets en d\'eveloppement les sch\'emas de contr\^ole de diff\'erents m\'ecanismes en utilisant des outils de mod\'elisation \`a base de r\'eseaux de P\'etri, de machines \`a \'etat ou de GRAFCET.
  Les \'etudiants devaient analyser un cahier des charges li\'e \`a une maquette de manipulation et mod\'eliser le sch\'ema de contr\^ole \`a mettre en place.
  Ils devaient ensuite coder leur mod\'elisation selon diff\'erentes mises en place dans un langage d\'ependant de la maquette, ST ou IL pour un automate programmable, C pour un microcontr\^oleur ou VHDL pour un FPGA.\\
  %
  \textbf{Informatique Industrielle}\\
  Enseignement de 6h en TP pour les Licence 2 du d\'epartement EEA.\\
  L'objectif \'etait de mettre en place des notions d'informatique \`a travers l'analyse du codage des informations, le traitement de signaux binaires et num\'eriques et l'utilisation de ports d'entr\'ee et de sortie num\'eriques et analogiques.
  Les \'etudiants codaient en C des programmes permettant d'observer la diff\'erence de codage et d'interpr\'etation des informations num\'eriques.
  Ils d\'eveloppaient aussi des programmes permettant de r\'ecup\'erer et de rendre des informations \`a travers des ports d'entr\'ees et de sorties binaires et analogiques.\\
  %
  \textbf{Systèmes à Evenements Discrets}\\
  Enseignement de 10h en TP pour les Licence 1 du d\'epartement EEA.\\
  L'objectif était de mettre en place des notions de logique combinatoire et de logique séquentielle pour la représentation des systèmes à évènements discrets par une décomposition en "composants logiques".
  Les étudiants sont amenés à étudier différents systèmes à évènements discrets en réalisant une décomposition guidée des "composants" pour étudier le fonctionnement logique de chacuns.
  A travers les travaux pratiques, les étudiants manipulent les différents outils de la logique combinatoire (algèbre de Boole, tables de vérité et de Karnaugh, etc) ainsi que la mise en place de blocs de mémorisation (bascules, registres, etc) pour la description de mécanismes séquentiels tels que des compteurs.\\
  %
  \textbf{Base de l'architecture et des syst\`emes}\\
  Enseignement de 24h en TP pour les Licence 1 Informatique.\\
  L'objectif \'etait de donner aux \'etudiants des notions de logique binaire et combinatoire.
  Les \'etudiants devaient r\'ealiser des circuits logiques pour r\'ealiser diff\'erentes t\^aches.
  Ils devaient analyser le lien entre les entr\'ees, les \'etats du syst\`eme et les sorties afin de trouver les tables de v\'erit\'e et de d\'eduire les \'equations logiques pour r\'ealiser les circuits en simulation.
}
%
\cventrylong
{2019 -- 2021}
{Doctorant Contractuel Charg\'e d'Enseignement}
{Institut National des Sciences Appliqu\'ees de Toulouse (31)}
{D\'epartement de G\'enie Electrique et Informatique}
{}
{
  \textbf{Syst\`emes Temps R\'eel}\\
  Enseignement de 27,5h en TP pour les 4e ann\'ee AE-SE.\\
  L'objectif est de donner aux \'etudiants les notions de Syst\`emes Temps R\'eel.
  Les \'etudiants apprennent \`a mettre en place les notions vues en cours \`a travers la gestion de la communication avec une base mobile dont le contr\^ole a \'et\'e fourni int\'egrant des contraintes temps r\'eel sur les ordres re\c cus.\\
  %
  \textbf{Programmation microcontr\^oleur}\\
  Enseignement de 13,75h en TP pour les 4e ann\'ee AE-SE.\\
  L'objectif est de donner aux \'etudiants les notions de programmation en C sur des microcontr\^oleurs STM32.
  Les \'etudiants apprennent \`a mettre en place ces notions en d\'eveloppant le syst\`eme de contr\^ole d'un voilier.
  Ils devaient r\'ecup\'erer des informations transmises par une t\'el\'ecommande par des signaux RF et les analyser pour contr\^oler l'orientation de la voile ou de la gouverne.\\
  %
  \textbf{Programmation langage C}\\
  Enseignement de 12,5h en TD et 44h en TP pour les 3e ann\'ee IMACS et les 2\`eme ann\'ee MIC.\\
  L'objectif est de mettre en place les notions de programmation en langage C.
  Les \'etudiants sont amen\'es \`a \'ecrire des programmes permettant de d\'ecouvrir les diff\'erentes notions et syntaxes du C.
  Les \'etudiants sont ensuite amen\'es \`a appliquer ces notions dans la programmation d'un projet tel que la gestion d'une liste cha\^in\'ee repr\'esentant une course cycliste (IMACS).\\
  %
  \textbf{Approfondissement en Circuits Electroniques}\\
  Enseignement de 11h en TP pour les 3e ann\'ee IMACS.\\
  L'objectif est d'approfondir les connaissances en \'electronique analogique \`a travers l'utilisation de montages \`a transistors bipolaires.
  Les \'etudiants sont amen\'es \`a \'etudier le dimensionnement et \`a r\'ealiser le montage d'un pr\'e-amplificateur \`a base de transistors bipolaires.\\
  %
  \textbf{Concepts et Circuits pour le Traitement du Signal}\\
  Enseignement de 13,75h en TP pour les 2\`eme ann\'ee IMACS.\\
  L'objectif est d'\'etudier les montages \`a base d'Amplificateurs Op\'erationnels pour le traitement de signaux analogiques.
  Les \'etudiants sont ainsi amen\'es \`a dimmensionner, r\'ealiser et analyser diff\'erents montages \`a base d'Amplificateurs Op\'erationnels pour r\'ealiser diff\'erents traitement sur des signaux analogiques.\\
  %
  \textbf{Automatique Discr\`ete}\\
  Enseignement de 11h en TP pour les 2\`eme ann\'ee FAS.\\
  L'objectif est de donner des notions d'automatique discr\`ete \`a travers l'utilisation de diff\'erentes maquettes, telles qu'un ascenseur ou une montre, permettant de visualiser l'\'evolution d'un syst\`eme.\\
}
%

\section{Activit\'es de Recherche}
Mes activit\'es de recherches ont d\'ebut\'e en 4e ann\'ee AE-SE \`a l'INSA Toulouse dans le cadre d'un projet d'initiation \`a la recherche orient\'e sur la g\'en\'eration et le stockage d'\'energie.
Par la suite, je me suis orient\'e vers la robotique et plus pr\'ecis\'ement la vision par ordinateur pour les robots durant ma 5e ann\'ee \`a l'INSA.
J'aborde aujourd'hui les diff\'erents aspects n\'ecessaires pour am\'eliorer l'autonomie d'un robot, qui incluent l'utilisation de capteurs ext\'eroceptifs et particuli\`erement de capteurs LiDAR et de profondeur.\\
Mes recherches s'int\'eressent \`a la localisation d'objets, la relocalisation du robot dans son environnement en tout temps et la reconstruction de son environnement pour la planification de mouvements.
Mon int\'er\^et tourne aussi autour de l'embarquement de ces m\'ecanismes sur les capacit\'es de calcul du robot afin de r\'eduire la d\'ependance du robot \`a des plateformes de calcul externes.\\
\\
%
\cventrylong
{2019-2023}
{Th\`ese en Robotique}
{Doctorat r\'ealis\'e sous la direction d'Olivier Stasse}
{Reconnaissance d'objets visant \`a la locomotion et \`a la manipulation par un robot humano\"ide dans un environnement industriel}
{}
{
  Cette th\`ese s'applique au contexte de l'industrie a\'eronautique.
  Elle a pour cadre deux projets.
  Le premier est Robotics For the Future of Aircraft Manufacturing (ROB4FAM).
  Il s'agit d'un laboratoire joint entre Airbus Operations et l'\'equipe Gepetto du LAAS-CNRS.
  Il a pour but d'\'etudier la g\'en\'eration r\'eactive de mouvements robotiques pour des t\^aches de per\c cage et d'\'ebavurage destin\'ees \`a l'industrie a\'eronautique.
  Le second est le projet Europ\'een H2020 Memory of Motion (Memmo) coordonn\'e par l'\'equipe Gepetto et dont Airbus Operations est \'egalement partenaire.
  Ce projet a pour but de d\'evelopper des m\'ethodes pour g\'en\'erer des mouvements r\'eactifs et complexes ind\'ependamment de l'architecture du robot.
  Il se base pour cela sur une perception \'etendue de l'environnement et un apprentissage pr\'eliminaire des possibles mouvements du robot.
  Historiquement, l'\'equipe Gepetto travaille sur les robots humano\"ides, car ils repr\'esentent un challenge scientifique n\'ecessitant de d\'evelopper de nouveaux concepts.\\
  %
  La perception de l'environnement, quant \`a elle, peut se d\'ecouper en 4 grands axes : savoir o\`u sont nos outils, savoir o\`u on est, savoir o\`u on va, et savoir o\`u on met les pieds.
  Cette th\`ese vise \`a \'etudier les solutions qui peuvent \^etre int\'egr\'es dans un robot humano\"ide afin de r\'esoudre les probl\'ematiques mentionn\'ees.
  Ce travail s'appuie sur le robot humano\"ide Talos pour percevoir l'environnement en utilisant des donn\'ees issues de son LiDAR.\\
  %
  Ces travaux montrent premi\`erement qu'il est possible d'estimer pr\'ecis\'ement la position du robot dans son environnement {[2]}.
  Ceci est obtenu en utilisant des donn\'ees LiDAR et par l'int\'egration d'un syst\`eme d\'evelopp\'e par l'Universit\'e d'Oxford et fourni dans le cadre du projet Memmo.
  Le syst\`eme fourni est adjoint \`a une odom\'etrie visuelle-inertielle.\\
  %
  De plus, la r\'esolution du probl\`eme du robot kidnapp\'e est explor\'ee gr\^ace aux informations du LiDAR, de descripteurs g\'eom\'etriques et un m\'ecanisme de dictionnaire {[1]}.
  Ce probl\`eme se r\'esume \`a reconna\^itre l'environnement qui nous entoure \`a l'initialisation de la localisation du robot.
  Un syst\`eme est propos\'e pour r\'esoudre ce probl\`eme avec une pr\'ecision suffisante pour initialiser le syst\`eme de localisation pr\'ec\'edent.\\
  %
  Ensuite, la localisation d'objets volumineux \`a longue distance, tels que trouvable dans l'industrie a\'eronautique, est \'etudi\'ee.
  Cette localisation est r\'efl\'echie en utilisant les donn\'ees 3D et des descripteurs g\'eom\'etriques de type Fast Point Feature Histogram.\\
  L'\'etude est \'etendue aux r\'eseaux de neurones entra\^in\'es pour localiser des objets portant peu d'informations textuelles \`a travers l'utilisation de m\'ethodes issues de l'\'etat de l'art.\\
  %
  Enfin, un syst\`eme d\'etectant les plans au sol, utilisant un capteur d\'edi\'e ajout\'e au bassin du robot, est int\'egr\'e pour rendre le robot capable de planifier ses pas en ligne.
}
%
\cventrylong
{2019}
{Projet de Fin d'Etudes}
{Stage r\'ealis\'e au LAAS-CNRS sous la supervision d'Olivier Stasse et Fr\'ed\'eric Lerasle}
{Reconnaissance et localisation d'objets par vision embarqu\'ee (cam\'eras RGB ou RGB-D) sur un robot humano\"ide pour des t\^aches de vissage et de per\c cage}
{}
{
  Dans le cadre de leurs recherches, les roboticiens de l'\'equipe Gepetto ont cherch\'e \`a am\'eliorer les capacit\'es de manipulation de leurs robots.
  Afin de pouvoir manipuler des objets, le robot a besoin de rep\'erer l'objet dans son espace.
  C'est dans ce contexte que j'ai cherch\'e \`a utiliser des cam\'eras RGB-D et une premi\`ere architecture d\'efinie dans l'\'equipe pour localiser des objets.\\
  Les objectifs fix\'es \'etaient de prendre un objet dont nous n'avons pas de mod\`ele, d'avoir une phase de mod\'elisation pour obtenir une repr\'esentation de l'objet et d'utiliser cette mod\'elisation pour localiser l'objet sur une table devant le robot.
  Le syst\`eme de localisation de l'objet a pour objectif de fonctionner sur un ordinateur embarqu\'e sur le robot pour am\'eliorer son autonomie.\\
  La mod\'elisation est r\'ealis\'ee en utilisant un ensemble de prises de vues RGB-D de l'objet pour reconstruire un mod\`ele sous la forme d'un nuage de point.
  Dans le syst\`eme de localisation, la donn\'ee transmise par la cam\'era est d'abord pr\'e-trait\'ee afin de limiter l'espace de recherche et de d\'etecter les diff\'erents objets sur une table.
  Ensuite, le nuage de points mod\'elis\'e est compar\'e aux diff\'erents objets \`a travers des m\'ethodes de r\'ealignement de nuages de points bas\'ees sur l'\'etat de l'art pour d\'efinir la position de l'objet dans l'espace du robot.
}
%
\cventrylong
{2018}
{Projet d'Initiation \`a la Recherche}
{Projet r\'ealis\'ee \`a l'INSA Toulouse sous la supervision de Germain Garcia, Corinne Alonso et Kolja Neuhaus.}
{Etude d'un syst\`eme d'alimentation \'electrique d'\'electrolyseur pour fabriquer de l'hydrog\`ene \`a partir de CPV (solaire \`a concentration)}
{}
{
  Nous avons \'etudi\'e la faisabilit\'e d'un circuit \`a base de panneaux photovolta\"iques \`a concentration (CPV) pour alimenter un \'electrolyseur afin de produire de l'hydrog\`ene.
  L'hydrog\`ene est un vecteur d'\'energie propre et durable.
  Sa production \`a partir d'un \'electrolyseur est efficace, mais elle poss\`ede une limitation principale : la r\'eaction est lente \`a initier.
  L'\'electrolyseur n\'ecessite donc d'\^etre aliment\'e par une source suffisante et stable.
  Nos recherches se sont orient\'ees vers l'alimentation du syst\`eme de production avec des CPV et une batterie de support recharg\'e lors des pics d'alimentation et soutenant le syst\`eme lors des creux.
}
%

\section{Activit\'es d'Encadrement}
\cventry
{Avr. -- Sept. 2022}
{Dorian Baudu}
{Stagiaire}
{Dipl\^ome d'Ing\'enieur, majeure Syst\`eme Robotique et Drone}
{EFREI}
{
  Les objectifs \'etaient d'utiliser des m\'ethodes bas\'ees sur les r\'eseaux de neurones pour localiser des objets dans des images et d'int\'egrer le syst\`eme sur un robot pour que la localisation fonctionne en ligne et soit utilis\'ee dans des t\^aches de manipulation ou de locomotion.
}
%
\cventry
{Avr. -- Ao\^ut 2022}
{Weikang Zeng}
{Stagiaire}
{Master ISC - parcours MIR}
{Universit\'e Grenoble Alpes}
{
  Les objectifs \'etaient d'\'etudier le syst\`eme de localisation d'objets utilisant des donn\'ees LiDAR d\'evelopp\'e dans l'\'equipe et de proposer des am\'eliorations afin d'am\'eliorer la pr\'ecision du syst\`eme ainsi que le temps de calcul.
}
%
\cventry
{Mars -- Oct. 2021}
{Guillaume Gobin}
{Stagiaire}
{Master Intelligence Artificielle et Reconnaissance des Formes}
{Universit\'e Paul Sabatier Toulouse III}
{
  L'objectif \'etait d'\'etudier l'\'etat de l'art afin de proposer une solution pour le probl\`eme de reconnaissance de place pour un robot humano\"ide dans un environnement industriel en utilisant les donn\'ees g\'eom\'etriques et d'intensit\'e issues d'un LiDAR.
}
%
\cventry
{Juin -- Sept. 2020}
{Hugo Lefevre}
{Stagiaire}
{4e ann\'ee de formation Ing\'enieur, Sp\'ecialit\'e Automatique Electronique}
{INSA Toulouse}
{
  L'objectif \'etait de d\'evelopper des tests d'int\'egration pour un syst\`eme de SLAM utilis\'e dans l'\'equipe sur le robot humano\"ide TALOS en utilisant le simulateur Gazebo et une mod\'elisation de la salle d'exp\'erimentation du laboratoire.
}
%

\section{Activit\'e Collectives et Administratives}
Au cours de mes \'etudes et de mon doctorat, je me suis impliqu\'e \`a diff\'erents niveaux dans l'\'equipe Gepetto au LAAS-CNRS ainsi que dans l'Association Sportive de l'INSA.\\
Au sein de l'\'equipe Gepetto, j'ai \'et\'e co-responsable de l'organisation des bureaux de l'\'equipe ainsi que de l'accueil des nouveaux arrivants.
Je me suis assur\'e que chaque \'etudiant arrivant dans l'\'equipe se voit attribuer un lieu convenable pour la dur\'ee de son passage.
J'ai aussi particip\'e \`a l'int\'egration des nouveaux arrivants dans l'\'equipe.\\
Au sein de l'Association Sportive de l'INSA, j'ai particip\'e de 2016 \`a 2023 \`a l'entra\^inement des archers d\'ebutants afin de les initier \`a la pratique du tir \`a l'arc.
Durant cette p\'eriode, j'ai aussi particip\'e \`a l'organisation des d\'eparts en comp\'etitions et Championnats de France des comp\'etiteurs dont je faisais partie, en organisant les financements, transports et h\'ebergements n\'ecessaires.
Enfin, en ma qualit\'e de Tr\'esorier de l'AS de 2017 \`a 2018, j'ai particip\'e \`a l'organisation des multiples \'ev\'enements propos\'es par l'association ainsi qu'\`a l'organisation financi\`ere des d\'eplacements des sportifs des diff\'erentes sections.\\
%

\section{Publications Scientifiques}
% Underline my name
\toggletrue{myrefs}
\large{\underline{Articles} :}
\begin{itemize}%
\normalsize{\item[[1]\hspace{-2mm}]} \normalsize{\bibentry{lasguignes:irc:2023}}
\normalsize{\item[[2]\hspace{-2mm}]} \normalsize{\bibentry{lasguignes:icar:2021}}
\end{itemize}

\newsavebox\mytempbib
\savebox\mytempbib{\parbox{\textwidth}{\bibliography{cv}}}

%vspace*{0.3cm}
%\subsection{In submission}

\section{Activit\'es  Editoriales}
\large{\underline{Revues d'articles scientifiques} :}
\begin{itemize}%
\items IJRR, IEEE T-RO, IEEE RA-L, ICRA, IROS, Humanoids.
\end{itemize}

\section{Exp\'eriences Associatives}
\cventry
{Depuis 2016}
{Balma Arc Club}
{Membre de l'\'equipe DRE (depuis 2018), Assistant-entraineur (2016 -- 2020)}
{}
{}
{}
%
\cventry
{2017 -- 2023}
{Association Sportive de l'INSA de Toulouse}
{}
{}
{}
{
  \begin{itemize}
    \item Encadrant et membre de l'\'equipe comp\'etition de la section Tir \`a l'arc (2017 -- 2023)
    \item Tr\'esorier et responsable de la section Tir \`a l'arc (2017 -- 2018)
  \end{itemize}
}
%
\cventry
{2016 -- 2019}
{Toulouse Ing\'enierie Multidisciplinaire (TIM)}
{Membre du p\^ole \'electronique}
{}
{}
{
  \begin{itemize}
    \item Participation au d\'eveloppement du contr\^ole moteur \'electrique.
    \item Impl\'ementation d'un ``Battery Monitoring System''.
  \end{itemize}
}
%

\section{Centres d'inter\^et}

\cventry
{Tir \`a l'arc}
{Pratique en club et en comp\'etition}
{Participation aux Championnats de France Universitaire en tant que compétiteur de 2016 \`a 2020 et de 2022 \`a 2023, et en tant que coach en 2024 (ASUPS)}
{}
{}
{}
%
\cventry
{Informatique}
{Programmation de petits projets, jeux vid\'eos}
{Participation technique \`a des projets artistiques (afficheur \`a balayage, mini-serre automatis\'ee) ; programmation d'une gamelle automatis\'ee et programmable}
{}
{}
{}
%


\section{Projet d'enseignements}

Depuis le début de mes études suppérieures, je me suis intéressé à la transmission de la connaissance et à l'enseignement. Dans un premier temps dans ma pratique sportive, pendant laquelle j'ai pris part à l'encadrement de jeunes ou débutants au Balma Arc Club et au sein de l'Association Sportive de l'INSA. Puis, dès ma première année de doctorat, j'ai ajouté une dimensionnement d'enseignement à mon doctorat. J'ai ainsi pris part à l'enseignement au sein de l'INSA Toulouse puis de l'Université Toulouse III Paul Sabatier. De par ma formation originellement technique, débutée par un DUT GEII avant de rejoindre l'INSA, j'ai pu apporter un côté plus pratique à mes interventions dans divers domaines. Ainsi, j'ai pu intervenir aussi bien en électronique qu'en automatique ou en programmation. Pendant mes services d'enseignement, j'ai conservé une diversité qui me permet d'intervenir dans différentes matières, liées à ma formation tournée informatique industrielle, systèmes embarqués et automatique. Mes activités de recherche et l'équipe dans laquelle j'ai évolué m'ont aussi permis de développer des connaissances en robotique.

Au cours de mes enseignements à l'INSA pendant mon doctorat, je suis principalement intervenu dans l'encadrement des travaux dirigés et travaux pratiques, en m'impliquant dans les évaluations de certaines matières. J'ai ainsi pu observer les méthodes de constructions des enseignements. Par la suite, dans le cadre de mes contrats d'Attaché Temporaire d'Enseignement et de Recherche à l'Université et à l'INSA, j'ai eu l'opportunité d'être responsables d'enseignements. Dans ce cadre, j'ai apprécié les enseignements tels qu'ils étaient alors prodigués et fait évolué les enseignements en adaptant le contenu des cours, des exercices, des travaux pratiques et des évaluations. J'ai aussi eu l'occasion d'imaginer de nouveaux travaux pratiques pour apporter de nouveaux exemples aux étudiants. J'ai toujours cherché à apporter de nouvelles connaissances aux étudiants en avançant étape par étape, en commençant par des exercices guidés pour apprendre à monter une réflexion et ensuite étudier des travaux dans lequel ils sont moins guidés et doivent tracer eux-même leur réflexion pour arriver à un résultat fonctionnel et optimal. De plus, afin d'améliorer l'implication des étudiants dans les exercices, je me suis appliqué, et continuerai à m'appliquer, à avoir des exercices et des exemples pratiques, mélant des actions ludiques, professionnalisantes ou liant à des exemples de société pour ne pas appliquer simplement un principe mais voir que ce principe est déjà présent autour de nous. D'après moi, avoir de tels exemples aide à la compréhension des notions et au développement de certaines logiques.

Ma formation axée sur les systèmes embarqués m'a permis d'acquérir des connaissances en systèmes temps réels que j'ai ensuite entretenues et parfaites avec mes enseignements. Ces connaissances me permettront de former les étudiants sur les thématiques des systèmes temps réels, de la programmation parallèles et les problématiques environnantes. Mes compétences pourront servir aux enseignements dans le Master Ingénierie des Systèmes Temps Réels (ISTR) comme dans la formation Systèmes Robotiques et Interactifs (SRI) de l'UPSSITECH, voire dans la licence Électronique, Énergie électrique, Automatique (EEA) sur les thématiques d'Informatique Industrielle. En plus de la thématique temps réel visée par le poste, la diversité des enseignements que j'ai conservé au cours de mes services me permettra d'intervenir dans les différents domaines des systèmes embarqués. Je pourrais ainsi soutenir mes collègues et m'impliquer au besoin dans les enseignements en systèmes discrets et informatique industrielle. Enfin, de par mon doctorat en robotique, j'ai acquis des connaissances qui m'ont permis de donner des cours introductifs à la robotique. Ces mêmes connaissances pourraient être utilisées pour des enseignements en robotique. Dans ce dernier cadre, je pourrais m'impliquer pour la relance du master Automatique-Robotique (AURO) dans un effort commun avec mes collègues. Je pense qu'une formation, de type Master, orientée autour de la robotique reste nécessaire dans le paysage universitaire français et Toulousain. Dans ce cadre je suis prêt à m'impliquer dans les formations axées notamment sur le temps réel mais aussi sur la perception 3D ou d'autres enseignements où mes compétences peuvent être utiles.

Les différents enseignements sur lesquels je suis intervenu m'ont permis de voir différentes méthodes de pédagogie. Souvent axées autour de notions appliquées à des exercices, j'ai pu observer les avantages de certaines méthodes de travail, comme par exemple les travaux dirigés en groupes autonomes afin d'ouvrir les étudiants à tracer leur guide, sous la surveillance de l'encadrant. Au cours de mon cursus, j'ai eu l'occasion de rencontrer d'autres méthodes innovantes, telles que des classes inversées, de l'apprentissage par problèmes, de la pédagogie active et de l'autoévaluation régulière. Je pense personnellement que la pédagogie traditionnelle n'est pas adaptée à tous les enseignements ni à tous les étudiants, il est donc nécessaire de faire évoluer les méthodes pédagogiques pour que l'apprentissage soit plus efficace. Je suis ouvert à toutes les pédagogies qui pourront permettre aux étudiants à progresser et apprendre efficacement. Je suis aussi prêt à m'impliquer dans les discussions qui auraient lieu sur l'évolution des méthodes pédagogiques pour que le corps enseignant évolue ensemble et progresse, comme nos étudiants apprennent.

Dans le cadre de mes enseignements, j'ai aussi eu l'occasion de travailler avec des étudiants en situation atypiques, tels qu'un étudiant diagnostiqué TSA, une étudiante en situation de handicap moteur ou des étudiants dyslexiques, dysorthographiques ou autres. Pour la plupart, leurs situations atypiques ne présentaient de contraintes que par rapport aux méthodes d'enseignements et d'évaluation actuelles. Par ces expériences, et par une expérience plus personnelle avec une personne en situation de handicap, j'ai été sensibilisé à la problématique de l'inclusion dans les études suppérieures. J'ai essayé dans mes enseignements de m'adapter aux situations atypiques sans pour autant diminuer la qualité de l'enseignement et espère pouvoir m'impliquer encore dans l'inclusivité des formations et des locaux pour ne pas empêcher des étudiants à apprendre à cause de leur handicap.

\section{Projet de Recherche}

Dans les dernières décénies, la vision par ordinateur à pris de l'ampleur, notamment avec l'avèmenement de la recherche en robotique ou des progrès techniques dans les transports et notamment les véhicules. Les techniques de perception exteroceptives sont nécessaires pour s'adapter à un environnement variable et en évolution constante.
Parmi ces systèmes de perception, les systèmes de vision basés sur des caméras et des capteurs de profondeur ont gagné en attrait. En effet, ces derniers permettent d'avoir une représentation de l'environnement grâce aux textures visuelles ou à la géométrie.

La grandissante diversité de capteurs a permis d'abord d'imaginer combiner des caméras avec des capteurs proprioceptifs pour améliorer les systèmes de perceptions tels que les algorithmes de Simultaneous Localisation And Mapping (SLAM). Les progrès effectués dans les composants électroniques ont permis au fil des années de passer d'un LiDAR "point" à un LiDAR "plan" puis un capteur permettant d'obtenir en une mesure une représentation à large champs de vision de l'environnement. Dans la continuité de ces progrès, on voit aujourd'hui la naissance de capteurs mélant une technologie LiDAR et des caméras pour obtenir une représentation texturée visuellement et précise géométriquement. Ces capteurs permettent d'obtenir une meilleure représentation de l'environnement qui jusqu'alors était limité à l'une ou l'autre information ou à une distance courte. Grâce à ces progrès techniques dans les systèmes de capteurs, les algorithmes utilisés se complexifient et gagnent en fonctionnalité et en précision. Toutefois, le développement de ces algorithmes permet d'obtenir de meilleurs résultats mais augmente la difficulté d'embarquement dans des systèmes souvent limités en puissance de calcul et en énergie.

\subsection{La perception en robotique}

Pour mieux évoluer dans son environnement, un système autonome, tel un robot, peut présenter plusieurs besoin qui peuvent nécessiter un système de perception exteroceptif. Dans le cadre de mes recherches, j'ai résumé ces besoins en 3 grands axes:
\begin{itemize}
  \item La localisation du robot dans son environnement, pour pouvoir naviguer précisemment dans son environnement et savoir en tout temps où on est,
  \item La localisation d'objets, pour reconnaître des objets d'intérêt et pouvoir se diriger vers eux voire les manipuler,
  \item La localisation de surfaces, pour avoir une connaissance du sol et pouvoir l'utiliser pour se déplacer.
\end{itemize}

Au cours de ma thèse, j'ai étudié simplement chaque situation pour utiliser des alorithmes disponibles ou développer de nouveaux algorithmes afin de répondre aux différentes problématiques pour le robot Talos.

De par le contexte industriel qui était donné à mes travaux, avec des environnements composés de larges volumes avec peu de jeu de textures visuelles, je me suis attelé à l'utilisation de capteurs de profondeurs, et plus particulièrement de LiDARs. Ces capteurs de profondeurs permettent d'obtenir des données géométriques en 3 dimensions pour représenter l'environnement. Ces données sont ensuite utilisées pour extraire différentes informations de corrélation géométrique pour réaliser une tâche spécifique.

Je me suis ainsi adonné à chercher et implémenter des solutions pour permettre à un robot humanoïde de mieux évoluer dans son environnement en me référent parfois à une représentation connue de l'environnement, parfois à la découverte que l'on fait de cet environnement.

Je souhaite continuer ma recherche autour de la reconnaissance de l'environnement. Cette reconnaissance, particulièrement axée vers la segmentation des données mesurées pour découper les différents objets et les éléments de l'environnement. Cette segmentation présente plusieurs avantages. Les instances segmentées peuvent être ensuite utiliser pour localiser des objets d'intérêts pour des tâches de locomotion dirigée ou de manipulation. Elles présentent aussi un intérêt dans les systèmes de localisation puisque ces instances peuvent permettre d'amers dans des systèmes de localisation en donnant des points de référence dans l'environnement, comme l'étaient les plans extraits dans le système de localisation basé LiDAR implémenté pour Talos.

\subsection{L'embarquement des solutions}

Les algorithmes que j'ai utilisé et implémenté jusqu'alors permettent par exemple d'obtenir la localisation du robot dans son environnement ou une représentation des surfaces du sol pour pouvoir planifier les contacts utiles au déplacement du robot. Néanmoins, la complexité et le coût combinatoire de ces algorithmes ne permet pas d'imaginer les intégrer ensemble sur un robot tel qu'utilisé dans ma thèse sans faire appel à de nouvelles capacités de calcul voire à des plateformes de calcul externes.

Je souhaite inclure dans mes recherches un objectif de "frugalité" des algorithmes, mettant en adéquation les besoins applicatifs, les capteurs, les plateformes et les algorithmes. Ces recherches auront pour objectif premier de chercher à embarquer les algorithmes dans les systèmes robotiques.

En plus de la question de frugalité des systèmes de perception, pour un fonctionnement autonome et sécuritaire, il est nécessaire de s'assurer que les algorithmes de perception permettent d'obtenir des résultats en respectant des contraintes de latence.

L'embarquement des systèmes de perception recherchés demandera donc de respecter les contraintes temporelles impliquées par l'utilité et l'application ainsi que des contraintes matérielles.

\subsection{Intégration dans le laboratoire}

Ces travaux de recherches s'inscriront dans les travaux menés au Laboratoire d'Analyse et d'Architecture des Systèmes (LAAS-CNRS). Les recherches menées au LAAS-CNRS s'axent sur 4 grands axes, que sont l'Informatique, la Robotique, l'Automatique et les Micro et Nano Systèmes. Ces recherches visent à répondre aux grands défis à venir apportés par les progrès techniques dans les domaines tels que les transports, l'industrie, la santé ou les énergies. Mes travaux de recherches pourront particulièrement s'inscrire dans les travaux de l'équipe Robotique Action et Perception (RAP) du département Robotique dont les travaux s'axent sur la conception d'algorithmes pour la perception ainsi que la conception de systèmes mélant la perception et le contrôle par exemple pour le déplacement de robots dans des environnements agricoles. L'équipe pourrait notamment être intéréssée par mes recherches autour de la reconnaissance de l'environnement et l'utilisation de données 3D. De plus, l'équipe RAP s'est intéressée aux capteurs dits "intelligents", alliant des questions d'efficience énergétique, de faible latence, ainsi que les questions d'embarquement.

Au sein du bassin industriel Toulousain, de telles recherches peuvent par exemple intéresser les entreprises ayant des intérêts dans les mobilités nouvelles, telles qu'Easy Mile, ou l'évolution de l'Industrie, comme Airbus. Enfin, les travaux sur la reconnaissance de l'environnement peuvent aussi ouvrir à des collaborations internes par exemple avec l'équipe GEPETTO travaillant sur la génération de mouvement de locomotion et de manipulation.


% {\color{red} ajouter problèmes de temporalité des informations => contraintes du temps réel à respecter}

% Couper en deux parties:\\
% ~1 - Quels objectifs à la perception: \\
% ~~a. localisation simple du robot -> objectifs de navigation, point plutôt utilisateur!\\
% ~~b. découverte de l'environnement + localisation d'objet (potentiellement amers pour la localisation) -> objectif: reconnaissance d'obstacles, manipulation (loca précise) -> utiles pour certains projets actuels de RAP, peut ouvrir à collaboation avec des équipes comme Gepetto\\
% ~2 - Quels principes pour l'embarquement dans les robots:\\
% ~~a. Frugalité des algorithmes\\
% ~~b. Capteurs intelligents (loooooong terme mais peut ouvrir à collaboration)


% adéquation des algorithmes et des plateformes
% - adéquation avec les objectifs (besoin, environnement, ...)
% - adéquation avec la puissance de calcul (=> séparer de l'utilisation de plateformes externes, => contrôle de la puissance de calcul & de l'énergie nécessaire)
% - -> ouverture au co-design de capteurs (participation avec équipe ??? spécialisée dans l'électronique et les composants) pour l'embarqué.

% Notions temps réel:
% - nécessité d'obtenir des résultats stables, contraint en temps selon l'application.
% - principes de l'embarquement sur les robots => adéquation HW-SW

\end{document}
