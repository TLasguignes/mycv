%% start of file `template.tex'.
%% Copyright 2006-2013 Xavier Danaux (xdanaux@gmail.com).
%
% This work may be distributed and/or modified under the
% conditions of the LaTeX Project Public License version 1.3c,
% available at http://www.latex-project.org/lppl/.


\documentclass[11pt,a4paper,sans]{moderncv}         % possible options include font size ('10pt', '11pt' and '12pt'), paper size ('a4paper', 'letterpaper', 'a5paper', 'legalpaper', 'executivepaper' and 'landscape') and font family ('sans' and 'roman')

% moderncv themes
\moderncvstyle{classic}                             % style options are 'casual' (default), 'classic', 'oldstyle' and 'banking'
\moderncvcolor{blue}                                % color options 'blue' (default), 'orange', 'green', 'red', 'purple', 'grey' and 'black'
%\renewcommand{\familydefault}{\sfdefault}          % to set the default font; use '\sfdefault' for the default sans serif font, '\rmdefault' for the default roman one, or any tex font name
%\nopagenumbers{}                                   % uncomment to suppress automatic page numbering for CVs longer than one page

% character encoding
\usepackage[utf8]{inputenc}                         % if you are not using xelatex ou lualatex, replace by the encoding you are using
%\usepackage{CJKutf8}                               % if you need to use CJK to typeset your resume in Chinese, Japanese or Korean

% adjust the page margins
%\usepackage[scale=0.75]{geometry}
%\usepackage[top=1.9cm, bottom=1.9cm, left=1.9cm, right=1.9cm]{geometry}
\usepackage[scale=.85]{geometry}
%\setlength{\hintscolumnwidth}{2.cm}                % if you want to change the width of the column with the dates
%\setlength{\makecvtitlenamewidth}{10cm}            % for the 'classic' style, if you want to force the width allocated to your name and avoid line breaks.
                                                    % be careful though, the length is normally calculated to avoid any overlap with your personal info; use this at your own typographical risks...

%** WHY ?
\usepackage{xparse}
%**

\usepackage{bibentry}
\nobibliography*

\bibliographystyle{my_plain}

\usepackage{etoolbox}
\newtoggle{myrefs}
\newcommand{\myref}[1]{\iftoggle{myrefs}{\textbf{#1}}{#1}}

\name{Thibaud}{Lasguignes}
\title{PhD student in humanoid robotics and computer vision}            % optional, remove / comment the line if not wanted
\address{4 All\'ee Elise Deroche}{31400 Toulouse}{} % optional, remove / comment the line if not wanted; the "postcode city" and and "country" arguments can be omitted or provided empty
\phone[mobile]{+33~(6)~26~36~28~13}                  % optional, remove / comment the line if not wanted
% \phone[fixed]{+2~(345)~678~901}                    % optional, remove / comment the line if not wanted
% \phone[fax]{+3~(456)~789~012}                      % optional, remove / comment the line if not wanted
\email{thibaud.lasguignes@sfr.fr}                  % optional, remove / comment the line if not wanted
% \homepage{www.johndoe.com}                         % optional, remove / comment the line if not wanted
\extrainfo{27 y.o. -- Driving licence}                       % optional, remove / comment the line if not wanted
%\photo[5cm]{../figures/DSC00009.jpeg}                % optional, remove / comment the line if not wanted; '64pt' is the height the picture must be resized to, 0.4pt is the thickness of the frame around it (put it to 0pt for no frame) and 'picture' is the name of the picture file
% \quote{Some quote}                                 % optional, remove / comment the line if not wanted

\newcommand{\items}{\item \hspace{2mm}}

% to show numerical labels in the bibliography (default is to show no labels); only useful if you make citations in your resume
%\makeatletter
%\renewcommand*{\bibliographyitemlabel}{\@biblabel{\arabic{enumiv}}}
%\makeatother
%\renewcommand*{\bibliographyitemlabel}{[\arabic{enumiv}]}% CONSIDER REPLACING THE ABOVE BY THIS

% bibliography with mutiple entries
%\usepackage{multibib}
%\newcites{book,misc}{{Books},{Others}}
%----------------------------------------------------------------------------------
%            content
%----------------------------------------------------------------------------------
\begin{document}
%\begin{CJK*}{UTF8}{gbsn}                          % to typeset your resume in Chinese using CJK
%-----       resume       ---------------------------------------------------------

\makecvtitle
\vspace*{-1cm}

\section{Trainings}
\cventry{Since 2019}{Preparation of a PhD in Humanoid Robotics and Computer Vision}
{LAAS-CNRS, in~Toulouse (31)}
{INSA Toulouse}
{supervised by Olivier Stasse}
{``Object recognition for locomotion and manipulation with a humanoid robot in an industrial environment''.}
\cventry{2019}{INSA engineering diploma, Speciality Automatic Electronics}
{INSA Toulouse (31)}{}{}{oriented on Critical Embedded Computing Systems}
%
\cventry{2016}{DUT Electrical Engineering and Industrial Informatics (GEII)}{IUT `A' Paul Sabatier}{Toulouse (31)}{(major)}{}%
%
\vspace*{0.5cm}
\cventry{Language}{French}{Native language}{}{}{}
\cventry{}{English}{read, spoken, written}{}{}{}
%
\cventry{Informatics}{Softwares}{cmake, git, ROS, PCL, Open3D}{}{}{}
%
\cventry{}{Languages}{C/C++, python, bash, Matlab, VHDL, Assembly, LADDER}{}{}{}



\section{Professional Experiences}
\cventry{2022-2023}{``Attach\'e Temporaire d'Enseignement et de Recherche''}{University Toulouse III Paul Sabatier (31)}{}{}
{
Teaching in the Electronics, Electrical energy and Automation department of the Faculty of Sciences and Engineering for a total of 192heTD.
\begin{itemize}
  \item Practical Work: Mobile Robotics and use of ROS (M2)
  \item Courses, Tutorials et Practical Work: Real Time Systems (UPSSITECH, M1)
  \item Practical Work: Programming notions in C (M1)
  \item Courses: Multithread programming and cocurrent execution (M1)
  \item Practical Work: Implementation of Discrete Event Systems (M1)
  \item Practical Work: Discrete Event Automatics (L2 and L3)
  \item Practical Work: Industrial Informatics (L2)
  \item Practical Work: Basics of architecture and systems (logic, L1)
\end{itemize}
}
\cventry{Since 2019}{PhD}{LAAS-CNRS}{Toulouse (31)}
{``Object recognition for locomotion and manipulation with a humanoid robot in an industrial environment''}
{
This work is carried out in the framework of ROB4FAM, a joint laboratory between AIRBUS and the LAAS-CNRS.
The objective is to develop and apply systems using depth information to recognise its environment in order to improve the autonomy of robots in an industrial environment.
This depth information is extracted from LiDAR or RGB-D cameras.
Assuming that \emph{a priori} knowledge of the environment is provided (map of the environment, model of objects of interest), the work focuses on the problems to localise oneself in the environment and localise objects.
The object localisation is extended to long distance localisation.
The research is based on the use of point cloud realignment methods, such as \emph{Iterative Closest Point}, and descriptors that allow to estimate correspondences between the measurements and the models.
}
%
\cventry{2019 -- 2021}{``Doctorant Contractuel Charg\'e d'Enseignement''}{INSA Toulouse (31)}{}
{}{
  Teaching in the Electrical and Computer Engineering department of ``Institut National des Sciences Appliquées'' of Toulouse over two years, for a total of 128 hours.
  \begin{itemize}
    \item Practical Work: Programming notions in C (2nd and 3rd years)
    \item Practical Work: Development of a Real Time system for the management of a mobile base in an arena (4th year)
    \item Practical Work: Introduction of notions on ``Logical Control Systems'' (2nd year)
    \item Practical Work: Implementation of a control system for a sailing boat on a micro-controller (3rd year)
    \item Practical Work: Dimensioning and assembly of a pre-amplifier based on bipolar transistors (3rd year)
    \item Practical Work: Assembly of circuits based on Operational Amplifiers (2nd year)
  \end{itemize}
}
%
\cventry{2019}{Trainee}{LAAS-CNRS}{Toulouse (31)}
{``Object recognition and localisation by embedded vision (RGB or RGB-D cameras) on an humanoid robot for screwing and drilling tasks''}{
This work concerns the localisation of objects using RGB-D cameras.
The object are assumed to be placed on a table and their models are known.
The objective is therefore to use point cloud realignment methods to provide a robot with the pose of the object to be manipulated.
}
%
\cventry{2018}{Trainee}{Technology and Strategy}{M\"unchen, Germany}
{In-vehicle tester for a BMW - Bosch project on an automatic parking aid}{
  Conducting in-vehicle tests to evaluate the capabilities of the developed system, identify flaws and enable its improvement.
}
%
\cventry{2016}{Trainee}{``Laboratoire d'Ing\'enierie des Syst\'emes Biologiques et des Proc\'ed\'es''}{Toulouse (31)}{}
{Code analysis and optimisation in Matlab in order to accelerate a system for optimising microbiological experiments.
}

\section{Scientific Papers}

% Underline my name
\toggletrue{myrefs}

\large{\underline{Articles:}}

\begin{itemize}%
\normalsize{\item[[1]\hspace{-2mm}]} \normalsize{\bibentry{lasguignes:icar:2021}}
\end{itemize}

\newsavebox\mytempbib
\savebox\mytempbib{\parbox{\textwidth}{\bibliography{cv}}}

%vspace*{0.3cm}
%\subsection{In submission}

\section{Editorial Activities}
\large{\underline{Reviews of scientific articles:}}
\begin{itemize}%
\items IJRR, IEEE T-RO, IEEE RA-L, ICRA, IROS, Humanoids.
\end{itemize}

\section{Associative Experiences}
\cventry{Since 2016}{Balma Arc Club}{Member of the DRE team (since 2018), Assistant coach (2016 -- 2020)}{}{}{}
\cventry{2017 -- 2023}{``Association Sportive'' of INSA Toulouse}{Supervisor and member of the competition team of the Archery section, head of the Archery section and Treasurer (2017 -- 2018)}{}{}{}
\cventry{2016 -- 2019}{Toulouse Ing\'enierie Multidisciplinaire (TIM)}{Member of the Electronics team}{}{}
{
  \begin{itemize}
    \item Participation in the development of the electric motor control,
    \item Implementation of a ``Battery Monitoring System''.
  \end{itemize}
}

\section{Interests}

\cventry{Archery}{Club and competition practice}{Participation in the French University Championships from 2016 to 2020 and from 2022 to 2023}{}{}{}
\cventry{Informatics}{Programming of personal projects, video games}{Technical participation in artistic projects (scanning display, automated tiny greenhouse); programming of an automated and programmable cat feeder}{}{}{}

\end{document}
