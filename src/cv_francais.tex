%% start of file `template.tex'.
%% Copyright 2006-2013 Xavier Danaux (xdanaux@gmail.com).
%
% This work may be distributed and/or modified under the
% conditions of the LaTeX Project Public License version 1.3c,
% available at http://www.latex-project.org/lppl/.


\documentclass[11pt,a4paper,sans]{moderncv}         % possible options include font size ('10pt', '11pt' and '12pt'), paper size ('a4paper', 'letterpaper', 'a5paper', 'legalpaper', 'executivepaper' and 'landscape') and font family ('sans' and 'roman')

% moderncv themes
\moderncvstyle{classic}                             % style options are 'casual' (default), 'classic', 'oldstyle' and 'banking'
\moderncvcolor{blue}                                % color options 'blue' (default), 'orange', 'green', 'red', 'purple', 'grey' and 'black'
%\renewcommand{\familydefault}{\sfdefault}          % to set the default font; use '\sfdefault' for the default sans serif font, '\rmdefault' for the default roman one, or any tex font name
%\nopagenumbers{}                                   % uncomment to suppress automatic page numbering for CVs longer than one page

% character encoding
\usepackage[latin1]{inputenc}                         % if you are not using xelatex ou lualatex, replace by the encoding you are using
%\usepackage{CJKutf8}                               % if you need to use CJK to typeset your resume in Chinese, Japanese or Korean

% adjust the page margins
% \usepackage[scale=0.75]{geometry}
\usepackage[top=1.9cm, bottom=1.9cm, left=1.9cm, right=1.9cm]{geometry}
%\setlength{\hintscolumnwidth}{2.cm}                % if you want to change the width of the column with the dates
%\setlength{\makecvtitlenamewidth}{10cm}            % for the 'classic' style, if you want to force the width allocated to your name and avoid line breaks.
                                                    % be careful though, the length is normally calculated to avoid any overlap with your personal info; use this at your own typographical risks...

\usepackage[T1]{fontenc}
\usepackage{lmodern} % d'autres polices sont possibles
\usepackage[frenchb]{babel}

%** WHY ?
\usepackage{xparse}
%**

\usepackage{soul}
\setul{1pt}{.4pt}% 1pt below contents

\usepackage{bibentry}
\usepackage{makecell}                               % for multiline cells
\usepackage{wrapfig}                                % for wraping figures
\nobibliography*

\bibliographystyle{unsrt-fr}

\usepackage{pifont}             % list of dings available here: https://latex-tutorial.com/bullet-styles/
\usepackage{enumitem}
\usepackage{scrextend}
\setlist{nolistsep,label=\ding{245}}

% A custom version of the \cventry command that supports large itemized lists
% inside argument #7 (the custom cvitemize lists should be used!)
\newcommand*{\cventrylong}[7][.25em]{%
  \begin{tabular}{@{}p{\hintscolumnwidth}@{\hspace{\separatorcolumnwidth}}p{\maincolumnwidth}@{}}%
    \raggedleft\hintstyle{#2} &{%
        {\bfseries#3}%
        \ifthenelse{\equal{#4}{}}{}{, {\slshape#4}}%
        \ifthenelse{\equal{#5}{}}{}{, #5}%
        \ifthenelse{\equal{#6}{}}{}{, #6}%
    }%
  \end{tabular}%
  \begin{addmargin}[\hintscolumnwidth+\separatorcolumnwidth]{0em}%
    {\small#7}%
  \end{addmargin}%
  \par\addvspace{#1}}
% A custom version of the itemize environment that sets the appropriate left
% margin for use inside \cventylong
\newlist{cvitemize}{itemize}{1}
\setlist[cvitemize]{label=\labelitemi,%
leftmargin=\hintscolumnwidth+\separatorcolumnwidth+\labelwidth+\labelsep}

\usepackage{etoolbox}
\newtoggle{myrefs}
\newcommand{\myref}[1]{\iftoggle{myrefs}{\textbf{#1}}{#1}}

\name{Thibaud}{Lasguignes}
\title{Docteur en robotique humano\"ide et vision par ordinateur}            % optional, remove / comment the line if not wanted
\address{2 All\'ee Elise Deroche}{R\'esidence Lindbergh, G2-202}{31400 Toulouse} % optional, remove / comment the line if not wanted; the "postcode city" and and "country" arguments can be omitted or provided empty
\phone[mobile]{+33~(6)~26~36~28~13}                   % optional, remove / comment the line if not wanted
% \phone[fixed]{+2~(345)~678~901}                     % optional, remove / comment the line if not wanted
% \phone[fax]{+3~(456)~789~012}                       % optional, remove / comment the line if not wanted
\email{thibaud.lasguignes@sfr.fr}                     % optional, remove / comment the line if not wanted
\social[github]{TLasguignes}
\social[linkedin]{thibaud-lasguignes}
% \homepage{www.johndoe.com}                          % optional, remove / comment the line if not wanted
\photo[4cm]{./photo_2023.jpg}                % optional, remove / comment the line if not wanted; '64pt' is the height the picture must be resized to, 0.4pt is the thickness of the frame around it (put it to 0pt for no frame) and 'picture' is the name of the picture file
% \quote{Some quote}                                  % optional, remove / comment the line if not wanted

\newcommand{\items}{\item \hspace{2mm}}

% to show numerical labels in the bibliography (default is to show no labels); only useful if you make citations in your resume
%\makeatletter
%\renewcommand*{\bibliographyitemlabel}{\@biblabel{\arabic{enumiv}}}
%\makeatother
%\renewcommand*{\bibliographyitemlabel}{[\arabic{enumiv}]}% CONSIDER REPLACING THE ABOVE BY THIS

% bibliography with mutiple entries
%\usepackage{multibib}
%\newcites{book,misc}{{Books},{Others}}
%----------------------------------------------------------------------------------
%            content
%----------------------------------------------------------------------------------
\begin{document}
%\begin{CJK*}{UTF8}{gbsn}                          % to typeset your resume in Chinese using CJK
%-----       resume       ---------------------------------------------------------

\makecvtitle
\vspace*{-1cm}

\vspace*{0.5cm}
\cventry
{Langues}
{Fran\c cais}
{langue maternelle}
{}
{}
{}
%
\cventry
{}
{Anglais}
{Niveau B2}
{}
{}
{}
%
\cventry{Informatique}
{Logiciels}
{CMake, Git, ROS, PCL, Open3D}
{}
{}
{}
%
\cventry
{}
{Langage}
{C/C++, Python, Bash, Matlab, LaTeX, VHDL, Langage d'assemblage, LADDER}
{}
{}
{}
%
\section{Exp\'eriences Professionnelles}
\cventry
{Depuis 2023}
{Attach\'e Temporaire d'Enseignement et de Recherche}
{Institut National des Sciences Appliqu\'ees de Toulouse (31)}
{}
{}
{
  R\'ealisation d'enseignement au sein du d\'epartement de G\'enie Electrique et Informatique de l'INSA Toulouse sur un total de 192h \'equivalent TD.
}
%
\cventry
{2022~--~2023}
{Attach\'e Temporaire d'Enseignement et de Recherche}
{Universit\'e Paul Sabatier Toulouse III (31)}
{}
{}
{
  R\'ealisation d'enseignement au sein du d\'epartement Electronique, Energie \'electrique et Automatique de la Facult\'e Sciences et Ing\'enierie sur un total de 192h \'equivalent TD.
}
%
\cventry
{2019~--~2021}
{Doctorant Contractuel Charg\'e d'Enseignement}
{INSA de Toulouse (31)}
{}
{}
{
  R\'ealisation d'enseignements au sein du d\'epartement G\'enie Electrique et Informatique de l'Institut National des Sciences Appliqu\'ees de Toulouse sur deux ans, soit sur un total de 128 heures.
}
%
\cventry
{2019}
{Stagiaire}
{LAAS-CNRS}
{Toulouse (31)}
{"Reconnaissance et localisation d'objets par vision embarqu\'ee sur un robot humano\"ide"}
{
  D\'eveloppement d'un syst\`eme visant \`a la localisation d'objets en utilisant des cam\'eras RGB-D, pour fournir \`a un robot la pose de l'objet \`a manipuler en utilisant des m\'ethodes de r\'ealignement de nuages de points.
  Objets suppos\'es pos\'es sur une table et leurs mod\`eles connus.
}
%
%
\cventry
{2018}
{Stagiaire}
{Technology and Strategy}
{Munich, Allemagne}
{Testeur en v\'ehicule dans le cadre d'un projet BMW - Bosch sur une aide au parking automatique}
{
  Conduite de tests en v\'ehicules pour \'evaluer les capacit\'es du syst\`eme d\'evelopp\'e, relever les failles et permettre son am\'elioration.
}
%
%
\cventry
{2016}
{Stagiaire}
{Laboratoire d'Ing\'enierie des Syst\'emes Biologiques et des Proc\'ed\'es}
{Toulouse (31)}
{}
{
  Analyse de code et d\'eveloppement sous Matlab dans le but d'acc\'el\'erer un syst\'eme d'optimisation d'exp\'eriences microbiologiques.
}
%

\section{Formations}
\cventry
{2024}
{Qualification au corps de Ma\^itre de Conf\'erences}
{Section CNU 61 - G\'enie informatique, automatique et traitement du signal}
{}
{}
{}
%
\cventry
{2023}
{Doctorat en Robotique}
{INSA de Toulouse et LAAS-CNRS, \`a~Toulouse (31)}
{"Reconnaissance d'objets visant \`a la locomotion et \`a la manipulation par un robot humano\"ide dans un environnement industriel"}
{supervis\'e par Olivier Stasse}
{}
%
\cventry
{2019}
{Dipl\^ome d'ing\'enieur INSA, Sp\'ecialit\'e Automatique Electronique}
{INSA de Toulouse (31)}
{orient\'e Syst\`emes Informatiques Embarqu\'es Critiques}
{}
{
  "Reconnaissance et localisation d’objets par vision embarqu\'ee (cam\'eras RGB ou RGB-D) sur un robot humano\"ide pour des t\^aches de vissage et de per\c cage", supervis\'e par Olivier Stasse au LAAS-CNRS
}
%
\cventry
{2016}
{DUT G\'enie Electrique et Informatique Industrielle}
{IUT `A' Paul Sabatier}
{\`a Toulouse (31)}
{(major)}
{}
%
\cventry
{2014}
{Baccalaur\'eat Scientifique option math\'ematiques}
{Lyc\'ee G\'en\'eral et Technologique Jean-Fran\c cois Champollion}
{\`a Figeac (46)}
{(mention Bien)}
{}
%

\section{Publications Scientifiques}
% Underline my name
\toggletrue{myrefs}
\large{\ul{Th\`ese} :}
\begin{itemize}%
\normalsize{\item[[1]\hspace{-2mm}]} \normalsize{\bibentry{lasguignes:these:2023}}
\end{itemize}
\large{\ul{Articles} :}
\begin{itemize}%
\normalsize{\item[[1]\hspace{-2mm}]} \normalsize{\bibentry{lasguignes:irc:2023}}
\normalsize{\item[[2]\hspace{-2mm}]} \normalsize{\bibentry{lasguignes:icar:2021}}
\end{itemize}

\newsavebox\mytempbib
\savebox\mytempbib{\parbox{\textwidth}{\bibliography{cv}}}

%vspace*{0.3cm}
%\subsection{In submission}

\section{Activit\'es  Editoriales}
\large{\ul{Revues d'articles scientifiques} :}
\vspace{.1em}
\begin{itemize}
  \item[] IJRR, IEEE T-RO, IEEE RA-L, ICRA, IROS, Humanoids.
\end{itemize}

\section{Exp\'eriences Associatives}
\cventry
{Depuis 2016}
{Balma Arc Club}
{}
{}
{}
{
  \begin{itemize}
    \item Membre de l'\'equipe DRE (depuis 2018)
    \item Assistant-entraineur (2016~--~2020)
  \end{itemize}
}
%
\cventry
{2023~--~2024}
{Association Sportive de l'Universit\'e Paul Sabatier}
{}
{}
{}
{
  \begin{itemize}
    \item Encadrant du groupe comp\'etition de la section Tir \`a l'arc
  \end{itemize}
}
%
\cventry
{2017~--~2023}
{Association Sportive de l'INSA de Toulouse}
{}
{}
{}
{
  \begin{itemize}
    \item Encadrant et membre de l'\'equipe comp\'etition de la section Tir \`a l'arc (2017~--~2023)
    \item Tr\'esorier et responsable de la section Tir \`a l'arc (2017~--~2018)
  \end{itemize}
}
%
\cventry
{2016~--~2019}
{Toulouse Ing\'enierie Multidisciplinaire (TIM)}
{Membre du p\^ole \'electronique}
{}
{}
{
  \begin{itemize}
    \item Participation au d\'eveloppement du contr\^ole moteur \'electrique
    \item Impl\'ementation d'un ``Battery Monitoring System''
  \end{itemize}
}
%

\section{Centres d'inter\^et}

\cventry
{Tir \`a l'arc}
{Pratique en club et en comp\'etition}
{Participation aux Championnats de France Universitaire de 2016 \`a 2020 et de 2022 \`a 2023}
{}
{}
{}
%
\cventry
{Informatique}
{Programmation de petits projets, jeux vid\'eos}
{Participation technique \`a des projets artistiques (afficheur \`a balayage, mini-serre automatis\'ee) ; programmation d'une gamelle automatis\'ee et programmable}
{}
{}
{}
%
\end{document}
