%% start of file `template.tex'.
%% Copyright 2006-2013 Xavier Danaux (xdanaux@gmail.com).
%
% This work may be distributed and/or modified under the
% conditions of the LaTeX Project Public License version 1.3c,
% available at http://www.latex-project.org/lppl/.


\documentclass[11pt,a4paper,sans]{moderncv}         % possible options include font size ('10pt', '11pt' and '12pt'), paper size ('a4paper', 'letterpaper', 'a5paper', 'legalpaper', 'executivepaper' and 'landscape') and font family ('sans' and 'roman')

% moderncv themes
\moderncvstyle{classic}                             % style options are 'casual' (default), 'classic', 'oldstyle' and 'banking'
\moderncvcolor{blue}                                % color options 'blue' (default), 'orange', 'green', 'red', 'purple', 'grey' and 'black'
%\renewcommand{\familydefault}{\sfdefault}          % to set the default font; use '\sfdefault' for the default sans serif font, '\rmdefault' for the default roman one, or any tex font name
%\nopagenumbers{}                                   % uncomment to suppress automatic page numbering for CVs longer than one page

% character encoding
\usepackage[utf8]{inputenc}                         % if you are not using xelatex ou lualatex, replace by the encoding you are using
%\usepackage{CJKutf8}                               % if you need to use CJK to typeset your resume in Chinese, Japanese or Korean

% adjust the page margins
%\usepackage[scale=0.75]{geometry}
%\usepackage[top=1.9cm, bottom=1.9cm, left=1.9cm, right=1.9cm]{geometry}
\usepackage[scale=.85]{geometry}
%\setlength{\hintscolumnwidth}{2.cm}                % if you want to change the width of the column with the dates
%\setlength{\makecvtitlenamewidth}{10cm}            % for the 'classic' style, if you want to force the width allocated to your name and avoid line breaks.
                                                    % be careful though, the length is normally calculated to avoid any overlap with your personal info; use this at your own typographical risks...

%** WHY ?
\usepackage{xparse}
%**

\usepackage{bibentry}
\usepackage{makecell}
\nobibliography*

\bibliographystyle{my_plain}

\usepackage{etoolbox}
\newtoggle{myrefs}
\newcommand{\myref}[1]{\iftoggle{myrefs}{\textbf{#1}}{#1}}

\name{Thibaud}{Lasguignes}
\title{Docteur en robotique humano\"ide et vision par ordinateur}            % optional, remove / comment the line if not wanted
\address{4 All\'ee Elise Deroche}{R\'esidence Lindbergh, G1-407}{31400 Toulouse} % optional, remove / comment the line if not wanted; the "postcode city" and and "country" arguments can be omitted or provided empty
\phone[mobile]{+33~(6)~26~36~28~13}                  % optional, remove / comment the line if not wanted
% \phone[fixed]{+2~(345)~678~901}                    % optional, remove / comment the line if not wanted
% \phone[fax]{+3~(456)~789~012}                      % optional, remove / comment the line if not wanted
\email{thibaud.lasguignes@sfr.fr}                  % optional, remove / comment the line if not wanted
\social[github]{TLasguignes}
% \homepage{www.johndoe.com}                         % optional, remove / comment the line if not wanted
%\photo[5cm]{../figures/DSC00009.jpeg}                % optional, remove / comment the line if not wanted; '64pt' is the height the picture must be resized to, 0.4pt is the thickness of the frame around it (put it to 0pt for no frame) and 'picture' is the name of the picture file
% \quote{Some quote}                                 % optional, remove / comment the line if not wanted

\newcommand{\items}{\item \hspace{2mm}}

% to show numerical labels in the bibliography (default is to show no labels); only useful if you make citations in your resume
%\makeatletter
%\renewcommand*{\bibliographyitemlabel}{\@biblabel{\arabic{enumiv}}}
%\makeatother
%\renewcommand*{\bibliographyitemlabel}{[\arabic{enumiv}]}% CONSIDER REPLACING THE ABOVE BY THIS

% bibliography with mutiple entries
%\usepackage{multibib}
%\newcites{book,misc}{{Books},{Others}}
%----------------------------------------------------------------------------------
%            content
%----------------------------------------------------------------------------------
\begin{document}
%\begin{CJK*}{UTF8}{gbsn}                          % to typeset your resume in Chinese using CJK
%-----       resume       ---------------------------------------------------------

\makecvtitle
\vspace*{-1cm}

\section{Formations}
\cventry
{2023}
{Doctorat en Robotique}
{INSA de Toulouse et LAAS-CNRS, \`a~Toulouse (31)}
{"Reconnaissance d'objets visant \`a la locomotion et \`a la manipulation par un robot humano\"ide dans un environnement industriel"}
{supervis\'e par Olivier Stasse}
{
  Soutenue le 31 Ao\^ut 2023 devant le jury compos\'e de :\\
  \renewcommand{\arraystretch}{2}
  \begin{tabular}{l@{\hskip .5in}l@{\hskip .5in}l}
    Olivier Aycard         & \makecell{Ma\^itre de Conf\'erence\\Universit\'e Grenoble Alpes} & Rapporteur                \\
    David Filliat          & \makecell{Professeur\\ENSTA Paris}                               & Pr\'esident et Rapporteur \\
    Jean-Emmanuel Deschaud & \makecell{Charg\'e de Recherche\\Mines Paris}                    & Examinateur               \\
    Ariane Herbulot        & \makecell{Ma\^itre de Conf\'erence\\Universit\'e Paul Sabatier Toulouse III}  & Examinateur               \\
    Nicolas Mansard        & \makecell{Directeur de Recherche\\LAAS-CNRS}                     & Examinateur               \\
    Olivier Stasse         & \makecell{Directeur de Recherche\\LAAS-CNRS}                     & Directeur de Th\`ese      \\
  \end{tabular}
}
% {Ces travaux sont men\'es dans le cadre de ROB4FAM, un laboratoire commun entre AIRBUS et le LAAS-CNRS.
% L'objectif est de d\'evelopper et/ou appliquer des syst\`emes utilisant des informations extraites de LiDAR ou de cam\'eras RGB-D pour reconna\^itre son environnement, afin d'am\'eliorer l'autonomie des robots dans un milieu industriel.
% Dans l'hypoth\`ese o\`u des connaissances a priori de l'environnement (carte de l'environnement, mod\`ele des objets d'int\'er\^et) sont fournies, les travaux s'axent sur deux probl\`emes: se localiser dans l'environnement et localiser les objets utiles qu'ils soient proches ou \'eloign\'es.
% Ces recherches se basent sur l'utilisation de m\'ethodes de r\'ealignement de nuages de points, de type \emph{Iterative Closest Point}, et de descripteurs permettant d'estimer des correspondances entre nos connaissances et les mesures.
% }
%
\cventry
{2019}
{Dipl\^ome d'ing\'enieur INSA, Sp\'ecialit\'e Automatique Electronique}
{INSA de Toulouse (31)}
{orient\'e Syst\`emes Informatiques Embarqu\'es Critiques}
{}
{
  "Reconnaissance et localisation d’objets par vision embarqu\'ee (cam\'eras RGB ou RGB-D) sur un robot humano\"ide pour des t\^aches de vissage et de per\c cage", supervis\'e par Olivier Stasse au LAAS-CNRS
}
%
\cventry
{2016}
{DUT G\'enie Electrique et Informatique Industrielle}
{IUT `A' Paul Sabatier}
{\`a Toulouse (31)}
{(major)}
{}
%
\cventry
{2014}
{Baccalaur\'eat Scientifique option math\'ematiques}
{Lyc\'ee G\'en\'eral et Technologique Jean-Fran\c cois Champollion}
{\`a Figeac (46)}
{(mention Bien)}
{}
%
\vspace*{0.5cm}
\cventry
{Langues}
{Fran\c cais}
{langue maternelle}
{}
{}
{}
%
\cventry
{}
{Anglais}
{lu, \'ecrit, parl\'e}
{}
{}
{}
%
\cventry{Informatique}
{Logiciels}
{cmake, git, ROS, PCL, Open3D}
{}
{}
{}
%
\cventry
{}
{Langage}
{C/C++, python, bash, Matlab, VHDL, Langage d'assemblage, LADDER}
{}
{}
{}
%

\section{Exp\'eriences Professionnelles}
\cventry
{Depuis 2023}
{Attach\'e Temporaire d'Enseignement et de Recherche}
{Institut National des Sciences Appliqu\'ees de Toulouse (31)}
{}
{}
{
  R\'ealisation d'enseignement au sein du d\'epartement de G\'enie Electrique et Informatique de l'INSA Toulouse sur un total de 192h \'equivalent TD.
% \begin{itemize}
  % \item Travaux Pratiques: D\'eveloppement de Syst\`emes Temps R\'eel (niveau M1)
  % \item Travaux Pratiques avec Projet: Mise en place de notions de programmation orient\'ee object, langage C++ (niveau M1)
  % \item Travaux Dirig\'es et Travaux Pratiques: Mise en place de notions de programmation Langage C (niveau L3)
  % \item Travaux Dirig\'es et Travaux Pratiques: Algorithmique et programmation ADA (niveau L1 et L2)
  % \item Cours, Travaux Dirig\'es et Travaux Pratiques: Mise en place de notions de programmation Python (niveau L1)
% \end{itemize}
}
%
\cventry
{2022 -- 2023}
{Attach\'e Temporaire d'Enseignement et de Recherche}
{Universit\'e Paul Sabatier Toulouse III (31)}
{}
{}
{
  R\'ealisation d'enseignement au sein du d\'epartement Electronique, Energie \'electrique et Automatique de la Facult\'e Sciences et Ing\'enierie sur un total de 192h \'equivalent TD.
% \begin{itemize}
  % \item Travaux Pratiques: Robotique Mobile et utilisation de ROS (niveau M2)
  % \item Cours, Travaux Dirig\'es et Travaux Pratiques: Syst\`eme Temps R\'eel (UPSSITECH, niveau M1)
  % \item Travaux Pratiques: Mise en place de notions de programmation Langage C (niveau M1)
  % \item Cours: Programmation Multithread et ex\'ecution concurrente (niveau M1)
  % \item Travaux Pratiques: Mise en oeuvre de Syst\`emes \`a Ev\`enements Discrets (niveau M1)
  % \item Travaux Pratiques: Automatique \`a Ev\`enements Discrets (niveau L2 et L3)
  % \item Travaux Pratiques: Informatique Industrielle (niveau L2)
  % \item Travaux Pratiques: Base de l'architecture et des syst\`emes (logique, niveau L1)
% \end{itemize}
}
%
\cventry
{2019 -- 2021}
{Doctorant Contractuel Charg\'e d'Enseignement}
{INSA de Toulouse (31)}
{}
{}
{
  R\'ealisation d'enseignements au sein du d\'epartement G\'enie Electrique et Informatique de l'Institut National des Sciences Appliqu\'ees de Toulouse sur deux ans, soit sur un total de 128 heures.
  % \begin{itemize}
  %   \item Travaux Pratiques: Mise en place des notions de programmation Langage C (2\`eme et 3\`eme ann\'ees)
  %   \item Travaux Pratiques: D\'eveloppement d'un syst\`eme Temps R\'eel pour la gestion du d\'eplacement d'une base mobile dans une ar\`ene (4\`eme ann\'ee)
  %   \item Travaux Pratiques: Mise en place de notions sur les ``Syst\`emes de commande logique'' (2\`eme ann\'ee)
  %   \item Travaux Pratiques: D\'eveloppement d'un syst\`eme sur micro-contr\^oleur pour le contr\^ole d'un voilier (3\`eme ann\'ee)
  %   \item Travaux Pratiques: Dimensionnement et montage d'un pr\'e-amplificateur \`a transistors bipolaires (3\`eme ann\'ee)
  %   \item Travaux Pratiques: Montage de circuits \`a base d'Amplificateurs Op\'erationnels (2\`eme ann\'ee)
  % \end{itemize}
}
%
\cventry
{2019}
{Stagiaire}
{LAAS-CNRS}
{Toulouse (31)}
{"Reconnaissance et localisation d'objets par vision embarqu\'ee sur un robot humano\"ide"}
{
  D\'eveloppement d'un syst\`eme visant \`a la localisation d'objets en utilisant des cam\'eras RGB-D, pour fournir \`a un robot la pose de l'objet \`a manipuler en utilisant des m\'ethodes de r\'ealignement de nuages de points.
  Objets suppos\'es pos\'es sur une table et leurs mod\`eles connus.
}
%
%
\cventry
{2018}
{Stagiaire}
{Technology and Strategy}
{Munich, Allemagne}
{Testeur en v\'ehicule dans le cadre d'un projet BMW - Bosch sur une aide au parking automatique}
{
  Conduite de tests en v\'ehicules pour \'evaluer les capacit\'es du syst\`eme d\'evelopp\'e, relever les failles et permettre son am\'elioration.
}
%
%
\cventry
{2016}
{Stagiaire}
{Laboratoire d'Ing\'enierie des Syst\'emes Biologiques et des Proc\'ed\'es}
{Toulouse (31)}
{}
{
  Analyse de code et d\'eveloppement sous Matlab dans le but d'acc\'el\'erer un syst\'eme d'optimisation d'exp\'eriences microbiologiques.
}
%

\section{Activit\'es d'Enseignement}
\cventry
{Depuis 2023}
{Attach\'e Temporaire d'Enseignement et de Recherche}
{Institut National des Sciences Appliqu\'ees de Toulouse (31)}
{D\'epartement de G\'enie Electrique et Informatique}
{}
{
  \textbf{Syst\`emes Temps R\'eel}\\
  Enseignement de 22h en TP pour les 4\`eme ann\'ee AE.\\
  L'objectif est de donner aux \'etudiants les notions de Syst\`emes Temps R\'eel.
  Les \'etudiants apprennent \`a mettre en place les notions vues en cours \`a travers la gestion de la communication avec une base mobile dont le contr\^ole a \'et\'e fourni int\'egrant des contraintes temps r\'eel sur les ordres reçus.
  \\
  %
  \textbf{Programmation Orient\'ee Objets}\\
  Enseignement de 19,25h en TP pour les 4\`eme ann\'ee AE-SE.\\
  L'objectif est de donner des notions de programmation orient\'ee objets en langage C++ \`a travers des exercices guid\'es et un projet.
  Les \'etudiants sont d'abord ammen\'es \`a \'etudier les diff\'erentes notions des objets \`a travers la cr\'eation de classe, l'h\'eritage et le polymorphisme.
  Par la suite, ils leur est demand\'e de r\'ealiser le projet de leur choix autour d'un microcontr\^oleur ESP8266 et de diff\'erents capteurs et actionneurs dont ils ont besoin, en mettant en place les notions vues en cours pour structurer leur programme.\\
  %
  \textbf{Programmation langage C}\\
  Enseignement de 6,25h en TD et 11h en TP pour les 3\`eme ann\'ee IMACS.\\
  L'objectif est de mettre en place les notions de programmation en langage C.
  Les \'etudiants sont ammen\'es \`a \'ecrire des programmes permettant de d\'ecouvrir les diff\'erentes notions et syntaxes du C.\\
  %
  \textbf{Algorithmique et programmation ADA}\\
  Enseignement de 51,25h en TD et 43,5h en TP pour les 1\`ere et 2\`eme ann\'ee.\\
  L'objectif est de donner des notions de d\'eveloppement d'algorithme et de r\'ealisation en programmation ADA.\\
  %
  \textbf{Algorithmique et programmation Python}\\
  Enseignement de 3,75h en CM, 2,5h en TD et 30,25h en TP pour les 2\`eme ann\'ee FAS.\\
  L'objectif est de donner les notions de programmation et d'algorithmique aux \'etudiants \`a travers du d\'eveloppement en python.\\
  %
}
%
\cventry
{2022 -- 2023}
{Attach\'e Temporaire d'Enseignement et de Recherche}
{Universit\'e Paul Sabatier Toulouse III (31)}
{Facult\'e Sciences et Ing\'enierie}
{D\'epartement Electronique, Energie \'electrique et Automatique}
{
  \textbf{Robotique Mobile et utilisation de ROS}\\
  Enseignement de 24h en TP pour les Master 2 Automatique et Robotique.\\
  L'objectif \'etait d'apprendre la structuration des architectures robotiques autour du middleware ROS.
  Les \'etudiants apprenaient l'utilisation du middleware en simulation pour faire \'evoluer des robots dans un environnement en utilisant les syst\`emes disponibles.
  Ils devaient aussi d\'evelopper leur propre n\oe ud de contr\^ole permettant de diriger un robot mobile Tiago afin de suivre une cible d\'etect\'ee avec une cam\'era RGB.\\
  %
  \textbf{Techniques et Impl\'ementation de M\'ethodes Num\'eriques}\\
  Enseignement de 58h en TP pour les Master 1 du d\'epartement EEA.\\
  L'objectif \'etait d'apprendre la programmation en langage C en d\'eveloppant des codes appliquant des m\'ethodes num\'eriques.
  Les \'etudiants d\'eveloppaient en C des programmes allant de la r\'esolution d'\'equation du second degr\`es aux int\'egrations par parties et interpolations lin\'eaires.\\
  %
  \textbf{Outils pour la Commande de Syst\`emes Parall\`eles}\\
  Enseignement de 10h en CM pour les Master 1 ISTR-AURO.\\
  L'objectif \'etait de donner aux \'etudiants les notions n\'ecessaire \`a la programmation parall\`eles en langage C.
  Les notions \'etudi\'ees allaient de la th\'eorie des processus et de la programation concurrente au partage de ressource et de donn\'ees.
  Les explications se voulaient g\'en\'eriques et utilisaient la librairie \emph{pthread} pour donner des exemples.
  Les \'etudiants ont ensuite \'et\'e ammen\'es en TD et en TP \`a \'etudier diff\'erent probl\`emes de partage de ressource et \`a mettre en place des solutions adapt\'ees en utilisant les notions vues en cours.\\
  %
  \textbf{Syst\`emes Temps R\'eel}\\
  Enseignement de 8h en CM, 10h en TD et 24h en TP pour les 2\`eme ann\'ee SRI \`a l'UPSSITECH.\\
  L'objectif \'etait de mettre en place les notions de syst\`eme temps r\'eel, d'ordonnancement de t\^aches et de criticit\'e des t\^aches.
  Les \'etudiants apprenaient \`a mettre en place les diff\'erentes notions \`a travers des analyses de cahier des charges pour mod\'eliser la commande r\'ealisable, l'analyse de code prenant en compte les temps de calculs pour v\'erifier l'ordonnancabilit\'e.
  Ils ont ensuite mis en pratique les notions vues en r\'ealisant un g\'en\'erateur de signaux sans notions de temps r\'eel et des g\'en\'erateurs de signaux avec diff\'erentes m\'ethodes appliquant les notions vues en cours et ont observ\'e les effets de l'ordonnancement dans le cas de syst\`eme surcharg\'es.\\
  %
  \textbf{Techniques de Mises en \OE uvre pour les Syst\`emes \`a Ev\`enements Discrets}\\
  Enseignement de 18h en TP pour les Master 1 ISTR-AURO.\\
  L'objectif \'etait de d\'evelopper les notions de mise en \oe uvre de syst\`emes \`a evenements distrets en partant d'une mod\'elisation pour aller sur la programmation.
  Les \'etudiants \'etaient ammen\'es \`a analyser un cahier des charges, mod\'eliser une commande r\'epondant et programmer la mod\'elisation en langage ST, IL, C ou VHDL selon la m\'ethode et la plateforme.
  Les \'etudiants ont ensuite dû d\'evelopper en C un syst\`eme automatisant le proc\'ed\'e de programation.
  Leurs programmes devaient r\'ecup\'erer les informations de la commande et \'ecrire le code n\'ecessaire \`a la mise en place du syst\`eme dans un langage d\'efini.\\
  %
  \textbf{Automatique \`a Evenements Discrets}\\
  Enseignement de 60h en TP pour les Licence 3 du d\'epartement EEA.\\
  L'objectif \'etait de mettre en place les notions d'automatique \`a \'ev\`enement discrets en d\'eveloppement les sch\'emas de controle de diff\'erents m\'ecanismes en utilisant des outils de mod\'elisation \`a base de r\'eseaux de P\'etri, de machines \`a \'etat ou de GRAFCET.
  Les \'etudiants devaient analyser un cahier des charges li\'es \`a une maquette de manipulation et mod\'eliser le sch\'ema de contr\^ole \`a mettre en place.
  Ils devaient ensuite coder leur mod\'elisation selon diff\'erentes mises en places dans un langage d\'ependant de la maquette, ST ou IL pour un automate programmable, C pour un microcontr\^oleur ou VHDL pour un FPGA.\\
  %
  \textbf{Informatique Industrielle}\\
  Enseignement de 6h en TP pour les Licence 2 du d\'epartement EEA.\\
  L'objectif \'etait de mettre en place des notions d'informatique \`a travers l'analyse du codage des informations, le traitement de signaux binaires et num\'eriques et l'utilisation de ports d'entr\'ee et de sortie num\'eriques et analogiques.
  Les \'etudiants codaient en C des programmes permettant d'observer la diff\'erence de codage et d'interpr\'etation des informations num\'eriques.
  Ils d\'eveloppaient aussi des programmes permettant de r\'ecup\'erer et rendre des informations \`a travers des ports d'entr\'ees et de sorties binaires et analogiques.\\
  %
  \textbf{Base de l'architecture et des syst\`emes}\\
  Enseignement de 24h en TP pour les Licence 1 Informatique.\\
  L'objectif \'etait de donner aux \'etudiants des notions de logique binaire et combinatoire.
  Les \'etudiants devaient r\'ealiser des circuits logiques pour r\'ealiser diff\'erentes t\^aches.
  Ils devaient analyser le lien entre les entr\'ees, les \'etats du syst\`eme et les sorties afin de trouver les table de v\'erit\'es et de d\'eduire les \'equations logiques pour r\'ealiser les circuits en simulation.
}
%
\cventry
{2019 -- 2021}
{Doctorant Contractuel Charg\'e d'Enseignement}
{Institut National des Sciences Appliqu\'ees de Toulouse (31)}
{D\'epartement de G\'enie Electrique et Informatique}
{}
{
  \textbf{Syst\`emes Temps R\'eel}\\
  Enseignement de 27,5h en TP pour les 4\`eme ann\'ee AE-SE.\\
  L'objectif est de donner aux \'etudiants les notions de Syst\`emes Temps R\'eel.
  Les \'etudiants apprennent \`a mettre en place les notions vues en cours \`a travers la gestion de la communication avec une base mobile dont le contr\^ole a \'et\'e fourni int\'egrant des contraintes temps r\'eel sur les ordres reçus.\\
  %
  \textbf{Programmation microcontrôleur}\\
  Enseignement de 13,75h en TP pour les 4\`eme ann\'ee AE-SE.\\
  L'objectif est de donner aux \'etudiants les notions de programmation en C sur des microcontrôleurs STM32.
  Les \'etudiants apprennent \`a mettre en place ces notions en développant le système de contrôle d'un voilier.
  Ils devaient récupérer des informations transmises par une télécommande par des signaux RF et les analyser pour contrôler l'orientation de la voile ou de la gouverne.\\
  %
  \textbf{Programmation langage C}\\
  Enseignement en TP et TD pour les 3\`eme ann\'ee IMACS et les 2ème année MIC.\\
  L'objectif est de mettre en place les notions de programmation en langage C.
  Les \'etudiants sont ammen\'es \`a \'ecrire des programmes permettant de d\'ecouvrir les diff\'erentes notions et syntaxes du C.
  Les étudiants sont ensuite ammenés à appliquer ces notions dans la programmation d'un projet tel que la gestion d'une liste chaînée représentant une course cycliste (IMACS).\\
  %
  \textbf{Approfondissement en Circuits Electroniques}\\
  Enseignement de 11h en TP pour les 3ème année IMACS.\\
  L'objectif est d'approfondir les connaissances en électronique analogique à travers l'utilisation de montages à transistors bipolaires.
  Les étudiants sont ammenés à étudier le dimensionnement et à réaliser le montage d'un pré-amplificateur à base de transistors bipolaires.\\
  %
  \textbf{Concepts et Circuits pour le Traitement du Signal}\\
  Enseignement de 13,75h en TP pour les 2ème année IMACS.\\
  L'objectif est d'étudier les montages à base d'Amplificateurs Opérationnels pour le traitement de signaux analogiques.
  Les étudiants sont ainsi ammenés à dimmensionner, réaliser et analyser différents montages à base d'Amplificateurs Opérationnels pour réaliser différents traitement sur des signaux analogiques.\\
  %
  \textbf{Automatique Discrète}\\
  Enseignement de 11h en TP pour les 2ème année FAS.\\
  L'objectif est de donner des notions d'automatique discrète à travers l'utilisation de différentes maquettes, telles qu'un ascenseur ou une montre, permettant de visualiser l'évolution d'un système.\\
}
%

\section{Activit\'es de Recherche}

\section{Activit\'es d'Encadrement}
\cventry
{Avr. -- Sept. 2022}
{Dorian Baudu}
{Stagiaire}
{}
{}
{
  Les objectifs \'etaient d'utiliser des m\'ethodes bas\'ees sur les r\'eseaux de neurones pour localiser des objets dans des images et d'int\'egrer le syst\`eme sur un robot pour que la localisation fonctionne en ligne et soit utilis\'e dans des t\^aches de manipulation ou de locomotion.
}
%
\cventry
{Avr. -- Ao\^ut 2022}
{Weikang Zeng}
{Stagiaire}
{Master ISC - parcours MIR}
{Universit\'e Grenoble Alpes}
{
  Les objectifs \'etaient d'\'etudier le syst\`eme de localisation d'objets utilisant des donn\'ees LiDAR d\'evelopp\'e dans l'\'equipe et de proposer des am\'eliorations afin d'am\'eliorer la pr\'ecision du syst\`eme ainsi que le temps de calcul.
}
%
\cventry
{Mars -- Oct. 2021}
{Guillaume Gobin}
{Stagiaire}
{Master Intelligence Artificielle et Reconnaissance des Formes}
{Universit\'e Paul Sabatier Toulouse III}
{
  L'objectif \'etait d'\'etudier l'\'etat de l'art afin de proposer une solution pour le probl\`eme de reconnaissance de place pour un robot humano\"ide dans un environnement industriel en utilisant les donn\'ees g\'eometriques et d'intensit\'e issues d'un LiDAR.
}
%
\cventry
{Juin -- Sept. 2020}
{Hugo Lefevre}
{Stagiaire}
{4\`eme ann\'ee de formation Ing\'enieur, Sp\'ecialit\'e Automatique Electronique}
{INSA Toulouse}
{
  L'objectif \'etait de d\'evelopper des tests int\'egratifs pour un syst\`eme de SLAM et utilis\'e dans l'\'equipe sur le robot humanoide TALOS en utilisant le simulateur Gazebo et une mod\'elisation de la salle d'exp\'erimentation du laboratoire.
}
%

\section{Publications Scientifiques}
% Underline my name
\toggletrue{myrefs}
\large{\underline{Articles} :}
\begin{itemize}%
\normalsize{\item[[1]\hspace{-2mm}]} \normalsize{\bibentry{lasguignes:irc:2023}}
\normalsize{\item[[2]\hspace{-2mm}]} \normalsize{\bibentry{lasguignes:icar:2021}}
\end{itemize}

\newsavebox\mytempbib
\savebox\mytempbib{\parbox{\textwidth}{\bibliography{cv}}}

%vspace*{0.3cm}
%\subsection{In submission}

\section{Activit\'es  Editoriales}
\large{\underline{Revues d'articles scientifiques} :}
\begin{itemize}%
\items IJRR, IEEE T-RO, IEEE RA-L, ICRA, IROS, Humanoids.
\end{itemize}

\section{Exp\'eriences Associatives}
\cventry
{Depuis 2016}
{Balma Arc Club}
{Membre de l'\'equipe DRE (depuis 2018), Assistant-entraineur (2016 -- 2020)}
{}
{}
{}
%
\cventry
{2017 -- 2023}
{Association Sportive de l'INSA de Toulouse}
{Encadrant et membre de l'\'equipe comp\'etition de la section Tir \`a l'arc, Tr\'esorier et responsable de la section Tir \`a l'arc (2017 -- 2018)}
{}
{}
{}
%
\cventry
{2016 -- 2019}
{Toulouse Ing\'enierie Multidisciplinaire (TIM)}
{Membre du p\^ole \'electronique}
{}
{}
{
  \begin{itemize}
    \item Participation au d\'eveloppement du contr\^ole moteur \'electrique.
    \item Impl\'ementation d'un ``Battery Monitoring System''.
  \end{itemize}
}
%

\section{Centres d'inter\^et}

\cventry
{Tir \`a l'arc}
{Pratique en club et en comp\'etition}
{Participation aux Championnats de France Universitaire de 2016 \`a 2020 et de 2022 \`a 2023}
{}
{}
{}
%
\cventry
{Informatique}
{Programmation de petits projets, jeux vid\'eos}
{Participation technique \`a des projets artistiques (afficheur \`a balayage, mini-serre automatis\'ee) ; programmation d'une gamelle automatis\'ee et programmable}
{}
{}
{}
%

\end{document}
