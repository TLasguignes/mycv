%% start of file `template.tex'.
%% Copyright 2006-2013 Xavier Danaux (xdanaux@gmail.com).
%
% This work may be distributed and/or modified under the
% conditions of the LaTeX Project Public License version 1.3c,
% available at http://www.latex-project.org/lppl/.


\documentclass[11pt,a4paper,sans]{moderncv}         % possible options include font size ('10pt', '11pt' and '12pt'), paper size ('a4paper', 'letterpaper', 'a5paper', 'legalpaper', 'executivepaper' and 'landscape') and font family ('sans' and 'roman')

% moderncv themes
\moderncvstyle{classic}                             % style options are 'casual' (default), 'classic', 'oldstyle' and 'banking'
\moderncvcolor{blue}                                % color options 'blue' (default), 'orange', 'green', 'red', 'purple', 'grey' and 'black'
%\renewcommand{\familydefault}{\sfdefault}          % to set the default font; use '\sfdefault' for the default sans serif font, '\rmdefault' for the default roman one, or any tex font name
%\nopagenumbers{}                                   % uncomment to suppress automatic page numbering for CVs longer than one page

% character encoding
\usepackage[utf8]{inputenc}                         % if you are not using xelatex ou lualatex, replace by the encoding you are using
%\usepackage{CJKutf8}                               % if you need to use CJK to typeset your resume in Chinese, Japanese or Korean

% adjust the page margins
%\usepackage[scale=0.75]{geometry}
%\usepackage[top=1.9cm, bottom=1.9cm, left=1.9cm, right=1.9cm]{geometry}
\usepackage[scale=.85]{geometry}
%\setlength{\hintscolumnwidth}{2.cm}                % if you want to change the width of the column with the dates
%\setlength{\makecvtitlenamewidth}{10cm}            % for the 'classic' style, if you want to force the width allocated to your name and avoid line breaks.
                                                    % be careful though, the length is normally calculated to avoid any overlap with your personal info; use this at your own typographical risks...

%** WHY ?
\usepackage{xparse}
%**

\usepackage{bibentry}
\nobibliography*

\bibliographystyle{my_plain}

\usepackage{etoolbox}
\newtoggle{myrefs}
\newcommand{\myref}[1]{\iftoggle{myrefs}{\textbf{#1}}{#1}}

\name{Thibaud}{Lasguignes}
\title{Doctorant en robotique humano\"ide et perception}            % optional, remove / comment the line if not wanted
\address{4 All\'ee Elise Deroche}{31400 Toulouse}{} % optional, remove / comment the line if not wanted; the "postcode city" and and "country" arguments can be omitted or provided empty
\phone[mobile]{+33~(6)~26~36~28~13}                  % optional, remove / comment the line if not wanted
% \phone[fixed]{+2~(345)~678~901}                    % optional, remove / comment the line if not wanted
% \phone[fax]{+3~(456)~789~012}                      % optional, remove / comment the line if not wanted
\email{thibaud.lasguignes@sfr.fr}                  % optional, remove / comment the line if not wanted
% \homepage{www.johndoe.com}                         % optional, remove / comment the line if not wanted
\extrainfo{27 ans -- Permis B}                       % optional, remove / comment the line if not wanted
%\photo[5cm]{../figures/DSC00009.jpeg}                % optional, remove / comment the line if not wanted; '64pt' is the height the picture must be resized to, 0.4pt is the thickness of the frame around it (put it to 0pt for no frame) and 'picture' is the name of the picture file
% \quote{Some quote}                                 % optional, remove / comment the line if not wanted

\newcommand{\items}{\item \hspace{2mm}}

% to show numerical labels in the bibliography (default is to show no labels); only useful if you make citations in your resume
%\makeatletter
%\renewcommand*{\bibliographyitemlabel}{\@biblabel{\arabic{enumiv}}}
%\makeatother
%\renewcommand*{\bibliographyitemlabel}{[\arabic{enumiv}]}% CONSIDER REPLACING THE ABOVE BY THIS

% bibliography with mutiple entries
%\usepackage{multibib}
%\newcites{book,misc}{{Books},{Others}}
%----------------------------------------------------------------------------------
%            content
%----------------------------------------------------------------------------------
\begin{document}
%\begin{CJK*}{UTF8}{gbsn}                          % to typeset your resume in Chinese using CJK
%-----       resume       ---------------------------------------------------------

\makecvtitle
\vspace*{-1cm}

\section{Formations}
\cventry{Depuis 2019}{Pr\'eparation d'un Doctorat en Robotique Humano\"ide et vision par ordinateur}
{LAAS-CNRS, \`a~Toulouse (31)}
{INSA de Toulouse}
{encadr\'e par Olivier Stasse}
{"Reconnaissance et localisation d'objets visant \`a la locomotion et la manipulation par un robot humano\"ide".}
\cventry{2019}{Dipl\^ome d'ing\'enieur INSA, Sp\'ecialit\'e Automatique Electronique}
{INSA de Toulouse (31)}{}{}{orient\'e Syst\`emes Informatiques Embarqu\'es Critiques}
%
\cventry{2016}{DUT G\'enie Electrique et Informatique Industrielle}{IUT `A' Paul Sabatier}{\`a Toulouse (31)}{(major)}{}%
%
\vspace*{0.5cm}
\cventry{Langue}{Français}{langue maternelle}{}{}{}
\cventry{}{Anglais}{lu, \'ecrit, parl\'e}{}{}{}
%
\cventry{Informatique}{Logiciels}{cmake, git, ROS, PCL, Open3D}{}{}{}
%
\cventry{}{Langage}{C/C++, python, bash, Matlab, VHDL, Langage d'assemblage, LADDER}{}{}{}



\section{Exp\'erience professionnelle}
\cventry{2022-2023}{Attach\'e Temporaire d'Enseignement et de Recherche}{Universit\'e Toulouse III Paul Sabatier (31)}{}{}
{
R\'ealisation d'enseignement au sein du d\'epartement Electronique, Energie \'electrique et Automatique de la Facult\'e Sciences et Ing\'enierie sur un total de 192h \'equivalent TD.
\begin{itemize}
  \item Travaux Pratiques: Robotique Mobile et utilisation de ROS (niveau M2)
  \item Cours, Travaux Dirig\'es et Travaux Pratiques: Syst\`eme Temps R\'eel (UPSSITECH, niveau M1)
  \item Travaux Pratiques: Mise en place de notions de programmation Langage C (niveau M1)
  \item Cours: Programmation Multithread et ex\'ecution concurrente (niveau M1)
  \item Travaux Pratiques: Mise en oeuvre de Syst\`emes \`a Ev\`enements Discrets (niveau M1)
  \item Travaux Pratiques: Automatique \`a Ev\`enements Discrets (niveau L2 et L3)
  \item Travaux Pratiques: Informatique Industrielle (niveau L2)
  \item Travaux Pratiques: Base de l'architecture et des syst\`emes (logique, niveau L1)
\end{itemize}
}
\cventry{Depuis 2019}{Doctorant}{LAAS-CNRS}{Toulouse (31)}
{"Reconnaissance et localisation d'objets visant \`a la locomotion et la manipulation par un robot humano\"ide"}
{Ces travaux sont men\'es dans le cadre de ROB4FAM, un laboratoire commun entre AIRBUS et le LAAS-CNRS.
L'objectif est de d\'evelopper et/ou appliquer des syst\`emes utilisant des informations extraites de LiDAR ou de cam\'eras RGB-D pour reconna\^itre son environnement, afin d'am\'eliorer l'autonomie des robots dans un milieu industriel.
Dans l'hypoth\`ese o\`u des connaissances a priori de l'environnement (carte de l'environnement, mod\`ele des objets d'int\'er\^et) sont fournies, les travaux s'axent sur deux probl\`emes: se localiser dans l'environnement et localiser les objets utiles qu'ils soient proches ou \'eloign\'es.
Ces recherches se basent sur l'utilisation de m\'ethodes de r\'ealignement de nuages de points, de type \emph{Iterative Closest Point}, et de descripteurs permettant d'estimer des correspondances entre nos connaissances et les mesures.
}
%
\cventry{2019 -- 2021}{Doctorant Contractuel Charg\'e d'Enseignement}{INSA de Toulouse (31)}{}
{}{
  R\'ealisation d'enseignements au sein du d\'epartement G\'enie Electrique et Informatique de l'Institut National des Sciences Appliqu\'ees de Toulouse sur deux ans, soit sur un total de 128 heures.
  \begin{itemize}
    \item Travaux Pratiques: Mise en place des notions de programmation Langage C (2\`eme et 3\`eme ann\'ees)
    \item Travaux Pratiques: D\'eveloppement d'un syst\`eme Temps R\'eel pour la gestion du d\'eplacement d'une base mobile dans une ar\`ene (4\`eme ann\'ee)
    \item Travaux Pratiques: Mise en place de notions sur les ``Syst\`emes de commande logique'' (2\`eme ann\'ee)
    \item Travaux Pratiques: D\'eveloppement d'un syst\`eme sur micro-contr\^oleur pour le contr\^ole d'un voilier (3\`eme ann\'ee)
    \item Travaux Pratiques: Dimensionnement et montage d'un pr\'e-amplificateur \`a transistors bipolaires (3\`eme ann\'ee)
    \item Travaux Pratiques: Montage de circuits \`a base d'Amplificateurs Op\'erationnels (2\`eme ann\'ee)
  \end{itemize}
}
%
\cventry{2019}{Stagiaire}{LAAS-CNRS}{Toulouse (31)}
{"Reconnaissance et localisation d'objets par vision embarqu\'ee sur un robot humano\"ide"}{
  Ces travaux concernent la localisation d'objets en utilisant des cam\'eras RGB-D.
Les objets sont suppos\'es pos\'es sur une table et leurs mod\`eles sont connus.
L'objectif est donc de pouvoir fournir \`a un robot la pose de l'objet \`a manipuler en utilisant des m\'ethodes de r\'ealignement de nuages de points.

}
%
\cventry{2018}{Stagiaire}{Technology and Strategy}{Munich, Allemagne}
{Testeur en v\'ehicule dans le cadre d'un projet BMW - Bosch sur une aide au parking automatique}{
  Conduite de tests en v\'ehicules pour \'evaluer les capacit\'es du syst\`eme d\'evelopp\'e, relever les failles et permettre son am\'elioration.
}
%
\cventry{2016}{Stagiaire}{Laboratoire d'Ing\'enierie des Syst\'emes Biologiques et des Proc\'ed\'es}{Toulouse (31)}{}
{Analyse de code et d\'eveloppement sous Matlab dans le but d'acc\'el\'erer un syst\'eme d'optimisation d'exp\'eriences microbiologiques.
}

\section{Publications Scientifiques}

% Underline my name
\toggletrue{myrefs}

\large{\underline{Articles :}}

\begin{itemize}%
\normalsize{\item[[1]\hspace{-2mm}]} \normalsize{\bibentry{lasguignes:icar:2021}}
\end{itemize}

\newsavebox\mytempbib
\savebox\mytempbib{\parbox{\textwidth}{\bibliography{cv}}}

%vspace*{0.3cm}
%\subsection{In submission}

\section{Activit\'es  \'Editoriales}
\large{\underline{Revues d'articles scientifiques :}}
\begin{itemize}%
\items IJRR, IEEE T-RO, IEEE RA-L, ICRA, IROS, Humanoids.
\end{itemize}

\section{Exp\'eriences Associatives}
\cventry{Depuis 2016}{Balma Arc Club}{Membre de l'\'equipe DRE (depuis 2018), Assistant-entraineur (2016 -- 2020)}{}{}{}
\cventry{2017 -- 2023}{Association Sportive de l'INSA de Toulouse}{Encadrant et membre de l'\'equipe comp\'etition de la section Tir \`a l'arc, Tr\'esorier et responsable de la section Tir \`a l'arc (2017 -- 2018)}{}{}{}
\cventry{2016 -- 2019}{Toulouse Ing\'enierie Multidisciplinaire (TIM)}{Membre du pôle \'electronique}{}{}
{
  \begin{itemize}
    \item Participation au d\'eveloppement du contr\^ole moteur \'electrique.
    \item Impl\'ementation d'un ``Battery Monitoring System''.
  \end{itemize}
}

\section{Centres d'inter\^et}

\cventry{Tir \`a l'arc}{Pratique en club et en comp\'etition}{Participation aux Championnats de France Universitaire de 2016 \`a 2020 et de 2022 \`a 2023}{}{}{}
\cventry{Informatique}{Programmation de petits projets, jeux vid\'eos}{Participation technique \`a des projets artistiques (afficheur \`a balayage, mini-serre automatis\'ee) ; programmation d'une gamelle automatis\'ee et programmable}{}{}{}


\end{document}
